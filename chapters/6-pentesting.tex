\chapter{Penetration testing} \label{ch:pentesting}
This chapter details all penetration tests that were performed on the system. These were derived from the threat model created in chapter \ref{ch:threat-model}. All penetration tests are described in the format outlined below. If an aspect of the test was considered simple then one or more parts have been omitted.
\begin{itemize}
    \item \textit{Introduction}. Describes the attack vector that this penetration test will explore.
    \item \textit{Background}. Details the required background knowledge to perform and evaluate this penetration test, if any.
    \item \textit{Method}. Describes in detail how the test was performed, e.g in what environment, with what tools, what commands were used, etc.
    \item \textit{Results}. Describes the findings of the penetration test.
    \item \textit{Discussion}. This section contains a discussion about the reliability, validity, and generalizability of the results.
\end{itemize}

\section{Task 1: Online Password Attack}
An online password attack refers to probing an active login, as opposed to an offline attack where you for example might try to crack a hash of a leaked database. This attack involves using some type of approach to test many passwords of a login form to try and find valid credentials.

\subsection{Background}
\todo

\subsection{Method}
\todo

\subsection{Results}
\todo

\subsection{Discussion}
\todo
