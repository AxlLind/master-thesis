\chapter{Penetration testing} \label{ch:pentesting}
This chapter details all penetration tests that were performed on the system. These were derived from the threat model created in chapter \ref{ch:threat-model}. All penetration tests are described in the format outlined below. If an aspect of the test was considered simple then one or more parts have been omitted.
\begin{itemize}
    \item \textit{Introduction}. Describes the attack vector that this penetration test will explore.
    \item \textit{Background}. Details the required background knowledge to perform and evaluate this penetration test, if any.
    \item \textit{Method}. Describes in detail how the test was performed, e.g in what environment, with what tools, what commands were used, etc.
    \item \textit{Results}. Describes the findings of the penetration test.
    \item \textit{Discussion}. This section contains a discussion about the reliability, validity, and generalizability of the results.
\end{itemize}

\section{Lab Environment} \label{ch:pentesting:lab-setup}
This section describes the lab environment and physical setup used when performing the pentests described below.

For the pentests involving the RF communication between the devices of the system specific hardware is required. Specifically, a Software Defined Radio (SDR) is required to be able to receive and transmit RF signals. The SDR used in this report is the HackRF One\footnotelink{greatscottgadgets.com/hackrf/one/}{2021-05-17} from Great Scott Gadgets\footnotelink{greatscottgadgets.com}{2021-05-17}. This is a popular and relatively cheap SDR, costing around 350 USD. Additionally, an ANT 500 antenna was used since the HackRF One does not come with an antenna. As for the physical setup, the RF transmitting devices of the system were placed relatively close to the SDR, within \texttt{10-20 cm}. This was done to increase the quality of the captures. By placing them close clean signals could be captured without increasing the gain parameters of the SDR. Figure \ref{fig:rf-lab-setup} shows a picture of the physical lab environment.
\begin{figure}[!ht]
    \centering
    \includegraphics[width=0.9\textwidth]{images/6-pentesting/lab-setup.png}
    \caption{The lab setup used to capture and replay RF signals.}
    \label{fig:rf-lab-setup}
\end{figure}

\section{Task 1: Replay attack on the RF communication} \label{ch:pentesting:replay}
This section details a penetration test against the RF communication between the hardware devices of the system. The specific attack vector explored in this section is a replay attack.

\subsubsection{Background}
A replay attack is an attack in network communication. The attacker listens in on the network traffic and simply retransmits the whole packets or information discovered in the packets to perform authenticated requests against a system they would otherwise not be able to. This type of vulnerability can have devastating consequences since even if several security measures have been taken, such as encrypting the data, the system can still be vulnerable to replay attacks. A replay attack is what is known as a \enquote{zero knowledge attack}, meaning the attacker needs no knowledge of how the message is structured or what it contains, and can even work on encrypted protocols \cite{hacking-the-iot-talk, rf-exploitation-talk}. This makes it often a very easy attack to perform.

Remote Keyless Entry (RKE) systems are notorious for being vulnerable to replay attacks. A famous example is RKE systems in car keys, used to unlock a car with from a distance a press of a button. When these first arrived on the market the security was extremely lacking and researchers in \citeyear{car-rke-systems} found that many cars on the market have used these insecure systems for over 20 years \cite{car-rke-systems}. Often one-way communication from the key to the car over RF signals is used with no protection at all against replay attacks. Simply capturing the RF signal and replaying it would unlock the car, allowing an attacker free access whenever they please. A simple and well-tested protection against replay attacks is sending time-stamps along with your message. If the timestamp is older than some threshold when the receiver would reject the message \cite{rke-replay}. This can, however, sometimes be difficult to implement in embedded systems since it requires both parties to agree on the time. For devices with an internet connection, this is easy thanks to the Network Time Protocol (NTP)\footnotelink{https://tools.ietf.org/html/rfc5905}{2021-05-03}. However, most low-powered simple IoT devices, such as a car key, do not have internet connectivity. A solution many modern systems used is a rolling code scheme \cite{hacking-the-iot-talk, kamkar2015drive}. Both systems keep an initially synchronized internal code $C$. When the transmitter sends a signal it includes this number and increments it internally. The receiver stores the last such number it received and checks that the number in the new signal is within some interval $[C+1, C+K]$, for some constant $K$. One usually accepts an interval of codes from the last accepted one in case a signal is lost or the button is pressed when you are outside the range of the car. If it is not within that range then the message is rejected. In practice, one might use this number to encrypt the message or use a sequence from a pre-defined pseudo-random number generator instead of an integer $C$.

This technique protects against replay attacks, however, it is still vulnerable against what is called a \textit{rolljam attack}. By jamming the signal and at the same time recording it the attacker now has a signal they can replay to have one-time access to unlock the system. Most modern cars are still susceptible to this type of attack \cite{hacking-the-iot-talk, kamkar2015drive}.

\subsubsection{Method} \label{ch:pentesting:replay:method}
The physical setup used during this test is described in section \ref{ch:pentesting:lab-setup}. As stated, a HackRF One SDR was used to receive and transmit RF signals. While low-level protocols exist to control the HackRF One, one usually uses higher-level tools to interact with it. The method of this test builds on the open-source tool \textit{Universal Radio Hacker} (URH), which was created by researchers at Hochschule Stralsund – University of Applied Sciences in Germany \cite{urh-paper}.

Initially, the center frequency at which the system communicates had to be found. This was done using the URH's \textit{Spectrum Analyzer} tool, see figure \ref{fig:finding-center-freq}. In the menu, \textit{HackRF} was selected under devices and the frequency band to listen to was selected. We know from the official documentation that the system communicates at \texttt{868 MHz} \citeyear{hsgw-user-manual}. Initially, that frequency was set in the menu options. The rest of the parameters were not important and left at their default values. To get the system to transmit data over RF communication, the tamper sensor located on the back of the door sensor was repeatedly pressed and released. While doing this, one could see a clear spike in the frequency spectrum and deduce that the center frequency used by the system is approximately \texttt{868.64 MHz}. This is clearly visible in figure \ref{fig:finding-center-freq}. The result makes sense since the frequency band \texttt{868.6–868.7 MHz} is allocated specifically for alarm systems in Europe \cite{etsi-srd-regulation}, as regulated by ETSI \footnotelink{https://www.etsi.org/}{2021-05-14}.
\begin{figure}[!ht]
    \centering
    \includegraphics[width=0.9\textwidth]{images/6-pentesting/find-frequency.png}
    \caption{Finding the center frequency of the RF communication.}
    \label{fig:finding-center-freq}
\end{figure}

When the center frequency had been found, signals could be captured using URH's \textit{Record Signal} tool, see figure \ref{fig:rf-signal-capture}. This tool records raw audio signals captures from an SDR. In the menu options, one has to select the correct frequency. After some trial and error, \texttt{868.638 MHz} was found to yield well-captured signals. By placing the HackRF close to the devices, the SDR was able to capture clean signals without increasing the gain parameters. See figure \ref{fig:rf-lab-setup} for the physical setup used during this test. By zooming into the captured audio signals and seeing clear sinusoidal waves with a non-trivial pattern, one could verify that the capture was most likely done correctly. Figure \ref{fig:zoomed-in-signal} shows a magnified view of a good signal captured from the door sensor's temper alarm. This process was repeated for all identified communication endpoints of the system, capturing their signals.
\begin{figure}[!ht]
    \centering
    \includegraphics[width=0.9\textwidth]{images/6-pentesting/signal-capture.png}
    \caption{Capturing an RF signal using Universal Radio Hacker.}
    \label{fig:rf-signal-capture}
\end{figure}
\begin{figure}[!ht]
    \centering
    \includegraphics[width=0.9\textwidth]{images/6-pentesting/zoomed-in-signal.png}
    \caption{A cleanly captured RF signal, showing a clear sinusoidal pattern.}
    \label{fig:zoomed-in-signal}
\end{figure}

Lastly, once signals had been captured they could be replayed using URH's \textit{Send Signal} tool, see figure \ref{fig:uhr-replay-tool}. The tool lets you easily select and edit which parts of the captured signal to replay. Through a trial and error process, several combinations of packets were tested as a replay attack.
\begin{figure}[!ht]
    \centering
    \includegraphics[width=0.9\textwidth]{images/6-pentesting/replay-signal.png}
    \caption{Performing a replay attack using Universal Radio Hacker.}
    \label{fig:uhr-replay-tool}
\end{figure}

\subsubsection{Results}
This test was successful and revealed \textit{critical} security flaws in the system. Every single tested endpoint of the RF communication was deemed vulnerable to replay attacks. The communication endpoints tested include the following:
\begin{itemize}
    \item \textbf{Arming and disarming} the alarm from the remote keypad.
    \item Triggering/resolving the tamper sensor of the door sensor.
    \item Triggering/resolving the tamper sensor of the camera.
    \item Triggering/resolving the tamper sensor of the main panel.
\end{itemize}
This has critical consequences for the security of the system. Capturing the RF signal of the user arming and disarming the alarm gives an attacker complete control of the arming state of the system. By simply replaying the signal, the main panel perceives this as a valid signal coming from the remote keypad.

\subsubsection{Discussion}
The system has no protection against replay attacks whatsoever. This conclusion is strengthened by the fact that replaying the same signals several \textit{months} later was still successful. However, replaying the signals against a different system of the same model was tested and it was unsuccessful. Presumably, some kind of ID of the device is sent making the packets invalid for another copy of the system. It clearly shows, however, that the manufacturer Climax Technology either lacks the knowledge and competence to implement secure RF communication or has decided not to prioritize security at the RF layer. Either way, this is a big mistake on their part which compromises the security of the whole system. SDRs have become much more available and much cheaper in the last few years. Attackers now have access to RF attacks which would have just a couple of years ago required very expensive and difficult to acquire hardware.

Furthermore, there are some factors that could potentially make this attack easier for an attacker in practice. When trying this attack signals were first incorrectly recorded at \texttt{868.0 MHz} instead of the correct frequency at \texttt{868.64 MHz}, giving rise to a very spiky signal. Still, the replay attack worked using this signal without issue. This indicates that perhaps the signal does not have to be captured that cleanly and that a badly captured signal, say from a distance or outside of the house, could be used to perform a replay attack. Presumably, the system implements some kind of error correction and noise filtering to increase the range and reliability of the RF communication. Additionally, regarding the tamper sensor signals, six signals are sent after one and the other, see figure \ref{fig:rf-signal-capture}. However, replaying only one of them still yields a successful replay attack. All of the six packets work individually. Presumably, the system repeats the same or similar signal several times for redundancy, in case one gets lost or corrupted in transit. This means, however, that an attacker only has to capture one of them to be able to perform a replay attack. The reliability of capturing and transmitting the signals from a distance is, however, a bit unclear. This was not tested. On the other hand, an attacker could easily purchase an RF amplifier to increase their range and sensitivity.

The consequences of this vulnerability are huge. An attack can completely bypass the alarm, defeating the whole purpose of the system in the first place. By replaying a captured disarm signal the attacker can disarm the system, granting them access to the property without triggering an alarm. Additionally, replaying any of the other signals, namely triggering the tamper sensor, an attacker could trigger an alarm of an armed system without actually entering the premises. The owner would then get a message of an active breach and security personnel would be sent to the site. One could imagine an attacker repeatedly doing this until the owner perhaps thinks the system is broken and uninstalls it, at least temporarily, from the premises. At which point the premises would no longer be protected and an attacker could strike.

\section{Task 2: Reverse engineering the RF protocol} \label{ch:pentesting:rf-reverse-engineering}
This section covers the process of reverse-engineering the RF protocol, via captured RF signals. This included mainly trying to demodulate the signals into binary data and then analyzing the contents.

\subsubsection{Background}
The system in question uses RF signals to wirelessly communicate across the devices. These radio waves are used to transfer binary data, ones and zeroes, between each device. Transferring binary data over radio waves is done by a process called modulation \cite{rf-modulation}. Digital modulation, e.g modulating binary data, is a whole field of study in itself, and that this is only a brief overview covering the basics of modulation.

There are three primary simple ways of modulating a binary signal: Amplitude- (ASK), Frequency- (FSK), and Phase- (PSK) shift keying. They each produce a distinct waveform, which can easily be visually identified. This is shown in figure \ref{fig:digital-modulation}.
\begin{figure}[!ht]
    \centering
    \includegraphics[width=0.5\textwidth]{images/6-pentesting/digital-modulation.png}
    \caption{The three primary techniques for digital modulation.}
    \label{fig:digital-modulation}
\end{figure}

Each technique uses a different property of the sinusoidal wave to encode a zero and one respectively. \gls{ASK} uses the amplitude of the wave, where a higher amplitude usually encodes a \texttt{1} and a lower amplitude a \texttt{0}. The frequency and phase of the wave is kept constant. \gls{FSK}, on the other hand, keeps both amplitude and phase constant. Instead two different frequencies are shifted between to differentiate between a \texttt{0} or \texttt{1}. Lastly, \gls{PSK} uses a phase-shift to differentiate between the two symbols. There are many much more complicated techniques to more efficiently encode a binary signal in radio waves, combining these techniques \cite{rf-modulation}. This can allow for much more information dense modulation schemes, however, these are considered outside the scope of this thesis.

Conversely, \textit{demodulation} is the process of converting a modulated signal back into the original binary data. In most real-world applications of RF communication this is done directly in hardware, using specific radio receivers and circuitry to automatically convert the signal back into binary data \cite{rf-modulation}. Often this hardware also implements error correction and other techniques to increase reliability of the communication, as interference and other disturbances are unavoidable in the real world. Figure \ref{fig:bfsk-demodulator} shows a very simplified circuit that implements binary FSK (BFSK) demodulation, e.g FSK modulation using two parameter frequencies.
\begin{figure}[!ht]
    \centering
    \includegraphics[width=0.8\textwidth]{images/6-pentesting/bfsk-demodulator.png}
    \caption{A simplified BFSK demodulating circuit.}
    \label{fig:bfsk-demodulator}
\end{figure}

Given a BFSK signal, modulated using the frequencies $f_1$ and $f_2$, the circuit does the following. First, the signal is split and passed into two band-pass filters, tuned to the two respective parameter frequencies. Then the absolute value is applied to the signals, one of them is inverted, and they are added together. Lastly, to convert the result into the final binary wave, the signal is passed to the $sign()$ function. This circuit was inspired by material from the course \textit{EE123 Digital Signal Processing} at Berkeley EECS\footnotelink{https://sites.google.com/berkeley.edu/ee123-sp20/labs}{2021-05-18}, specifically Lab 4 which covers FSK demodulation. Demodulation can of course also be done in software, but doing so in real-time is quite CPU-intensive and consequently a lot less energy efficient. Moreover, for each modulation type, there are several tuneable parameters that are hardcoded in the hardware demodulation chip, such as the frequencies of the FSK modulation, which during reverse engineering have to be figured out.

\subsubsection{Method}
Initially, signals were captured according to the method described in the method part of section \ref{ch:pentesting:replay:method}, using Universal Radio Hacker. Carefully inspecting the signals in \gls{URH}, as shown in figure \ref{fig:zoomed-in-signal}, one can see that clearly \gls{FSK} modulation is used to modulate the signal. This conclusion is corroborated by the official user manual submitted to the FCC \cite{hsgw-user-manual}, as well as official test documentation submitted to the \textit{Ministry of Internal Affairs and Communications} (MIC) in Japan \cite{mic-test-report}, in which the modulation type is specified to be FSK.

Furthermore, one of \gls{URH}'s key features is automatic demodulation \cite{urh}. This was tested on all captured signals. However, it was unfortunately continuously unsuccessful to automatically find the correct parameters of the modulation. Instead, the signals were demodulated \textit{by hand} using the open-source audio processing tool \textit{Audacity}\footnotelink{https://www.audacityteam.org}{2021-05-17}. The method to do this was inspired by the circuit shown in figure \ref{fig:bfsk-demodulator}, as well as an excellent tutorial by hobby radio enthusiast Nick Oakman, showing how to demodulate an FSK signal using Audacity \cite{oakman-fsk}. This was done in the following steps:
\begin{enumerate}
    \item The raw signal data, as captured by URH, was imported into Audacity via the menu options \textit{File - Import - Raw Data}. URH saves the signal as raw IQ data of signed 8-bit bytes. By selecting the encoding \textit{Signed 8-bit PCM} with \textit{2 Channels (Stereo)}, the I and Q components of the data get separated and imported into two separate tracks. Using only one of them is enough to extract information in this case. As such the stereo track was split into two mono tracks in Audacity and the second one was then deleted. The rest of the import options do not matter.

    \item Secondly, the center frequency, as perceived by Audacity, was noted. This can be found by the menu options \textit{Analyze - Plot Spectrum}. Note that due to several features like the sample rate not being part of the raw signal data, this frequency might not equal the actual center frequency of \texttt{868.64 MHz}. This frequency was used in the steps below.

    \item A \textit{High-Pass Filter} was applied to the first track, and a \textit{Low-Pass Filter} to the second, with the \textit{Roll-off} value set to \texttt{48 dB}. The former essentially map a higher frequency to a higher amplitude in the output wave, and the latter does the opposite.

    \item Using Audacity's scripting language Nyquist\footnotelink{https://www.audacityteam.org/about/nyquist}{2021-05-17} the absolute value was applied to each track. The short Nyquist program \mintinline[style=emacs]{cl}{(s-abs *track*)} was used to do this. Then a low-pass filter was applied to both tracks, computing the envelope of the signal. Lastly, the track that originally had the low-pass filter applied to it was inverted. This means that the two tracks now correspond to where the original signal was of high frequency and low frequency, respectively.

    \item Next, the two tracks was mixed together into a new track, e.g the signals were added together.

    \item The mixed track was amplified as high as possible with clipping enabled, creating a binary signal. This created the final signal, the original binary signal, demodulated from the original signal.
    
    \item Lastly, the final binary wave was exported as raw data to a file. This was done by selecting the final track and using the menu options \textit{File - Export - Export Selected Audio}. In the subsequent file dialog "Other uncompressed files" was selected as the type, with no heading (e.g RAW), and signed 8-bit PCM encoding. This gives you the signal as raw binary amplitude data, represented as signed bytes.
\end{enumerate}
Figure \ref{fig:audacity-demodulation} shows the process of demodulating a signal in Audacity. Each track in the figure is the result of one of the steps in the above explanation. This process was applied to all captured signals. However, the result of this process is a binary wave, not the actual binary data. To extract the binary data a simple python program was created, see figure \ref{lst:extract-bits}. The program expects a list of x-axis offsets, where the first bit of the signal is. These values were found by hand, by manually looking at the plot that the program creates. The signal length and symbol length were also measured by hand, however, these are constant for all captured signals. Only the signal start offsets had to be manually derived for each signal, which was quite labour intensive.
\begin{figure}[!ht]
    \centering
    \includegraphics[width=\textwidth]{images/6-pentesting/audacity-demodulation.png}
    \caption{The process of demodulating a BFSK signal in Audacity.}
    \label{fig:audacity-demodulation}
\end{figure}
\begin{figure}[ht]
    \begin{minted}[fontsize=\footnotesize, frame=single]{python}
import matplotlib.pyplot as plt

SIGNAL_LEN, SYMBOL_LEN = 34000, 200
FILE, SIGNAL_OFFSETS = "door-signal.raw", [1800, 856160, ...]

with open(FILE, "rb") as f:
  signal = [b if b < 128 else b - 256 for b in f.read()]

for i, offset in enumerate(SIGNAL_OFFSETS):
  xs = range(offset, offset + SIGNAL_LEN, SYMBOL_LEN)
  plt.scatter(xs, [0 for _ in xs], c="red")
  bits = "".join(['1' if signal[x] > 0 else '0' for x in xs])
  print(f"Packet {str(i).ljust(2)} =", hex(int(bits, 2)))

plt.plot(signal)
plt.show()
    \end{minted}
    \caption{A program to extract bits from a binary wave and plot it.}
    \label{lst:extract-bits}
\end{figure}

\subsubsection{Results}
By using the Audacity and the method described, binary data was extracted from each recorded signal. This is presented in table \ref{tb:demodulated-data}, in hexadecimal form.
\begin{table}[!p]
    \centering
    \begin{tabularx}{\textwidth}{l}
        \hline
        \textbf{Door tamper sensor on} \\ \hline
        \texttt{0xaaaaaaaa29cd29cd0a000015d477e072b922530064} \\
        \texttt{0xaaaaaaaa29cd29cd0a0000028648b07e291d2ceecc} \\
        \texttt{0xaaaaaaaa29cd29cd0a0000280b9d2e1d2d2ca7f31c} \\
        \texttt{0xaaaaaaaa29cd29cd0a00002548c662f2feeea7fe22} \\
        \texttt{0xaaaaaaaa29cd29cd0a000019201db301398d538674} \\
        \texttt{0xaaaaaaaa29cd29cd0a00000806d6a5ee37481e2f76} \\
        \hline
        
        \textbf{Door tamper sensor off} \\ \hline
        \texttt{0xaaaaaaaa29cd29cd0a0000102366a5cb78d61c0d0c} \\
        \texttt{0xaaaaaaaa29cd29cd0a00000a3b2cb0867bf62aa616} \\
        \texttt{0xaaaaaaaa29cd29cd0a000028fe2271f089a9e8c984} \\
        \texttt{0xaaaaaaaa29cd29cd0a00001e23195bcbe8c65107ec} \\
        \texttt{0xaaaaaaaa29cd29cd0a00001913b1ee7e3448da1cf0} \\
        \texttt{0xaaaaaaaa29cd29cd0a000006f69dbb732deb2a120c} \\
        \hline
        
        \textbf{Camera tamper sensor on} \\ \hline
        \texttt{0x155555554539a539a14000034164758f44cfae66f1} \\
        \texttt{0x155555554539a539a14000034164758f44cfae66f1} \\
        \texttt{0x155555554539a539a14000034164758f44cfae66f1} \\
        \texttt{0x155555554539a539a14000034164758f44cfae66f1} \\
        \texttt{0x155555554539a539a14000034164758f44cfae66f1} \\
        \texttt{0x155555554539a539a14000034164758f44cfae66f1} \\
        \hline
        
        \textbf{Camera tamper sensor off} \\ \hline
        \texttt{0x155555554539a539a140000342724d66fce053d3d7} \\
        \texttt{0x155555554539a539a140000342724d66fce053d3d7} \\
        \texttt{0x155555554539a539a140000342724d66fce053d3d7} \\
        \texttt{0x155555554539a539a140000342724d66fce053d3d7} \\
        \texttt{0x155555554539a539a140000342724d66fce053d3d7} \\
        \texttt{0x155555554539a539a140000342724d66fce053d3d7} \\
        \hline
    \end{tabularx}
    \caption{Binary data extracted from demodulating RF signals.}
    \label{tb:demodulated-data}
\end{table}
Analyzing the data one can see a very clear structure emerging. All recorded signals decoded into exactly 170 bits and seemingly always have the following structure. See also figure \ref{fig:rf-message-structure}.
\begin{enumerate}
    \item A preamble of alternating ones and zeroes. This is a very common technique in RF protocol to notify of an incoming signal and to sync clock frequencies \cite{hacking-the-iot-talk}.
    \item A sequence of bytes that is constant for each device. E.g the door sensor will always send the same byte sequence, the camera another one, etc. This corresponds to a syncword, which is often found in RF protocols, used to determine the protocol type or from which device the message originates \cite{hacking-the-iot-talk}. Presumably, this is some kind of ID to tell the main panel from which peripheral this message is.
    \item A sequence of zero bits. Presumably, this is part of the syncword. This is also a common technique in RF protocols, to let the receiver know what protocol this is and exactly where the payload data starts \cite{hacking-the-iot-talk}.
    \item A sequence of seemingly random bits. Presumably, this is the payload of the message.
\end{enumerate}
\begin{figure}[!ht]
    \centering
    \includegraphics[width=\textwidth]{images/6-pentesting/rf-message-structure.png}
    \caption{The structure of a message in the proprietary RF protocol.}
    \label{fig:rf-message-structure}
\end{figure}
Given the final part of the message, which is seemingly random, and the fact that the replay attack presented in section \ref{ch:pentesting:replay} worked for any of the messages, we can conclude that some kind of encryption or obfuscation is used. For the tamper sensor on the door, the payload is seemingly random for every single message. For the camera sensor, however, the packets are always identical.

\subsubsection{Discussion}
The signals were successfully demodulated using the described method. Data extracted from this process shows a clear structure to the messages, giving further indication that the capturing and demodulation was done correctly. By visually inspecting the signals one could see that FSK modulation was used. Additionally, this is actually documented in the official user manual \cite{hsgw-user-manual}, meaning we can be sure that this conclusion was correct. Furthermore, the signals follow patterns and structure commonly used in RF protocols \cite{hacking-the-iot-talk}, such as starting with a preamble of alternating symbols and a syncword, which in this case seems to be some kind of ID to identify themselves in the protocol. This is discussed in section \ref{ch:related-work:hacking-iot}. All this indicates that the capturing and demodulation of the signals were done correctly.

However, the payload is clearly encrypted or at least obfuscated in some way. This is in line with the documentation provided by Climax Technology, in which they reference a \enquote{Private Encryption Method} used in the RF communication \cite{hsgw-user-manual}. What the actual method is and whether or not it is a cryptographically secure method is left unanswered. It could be the case that the encryption is quite weak. Without access to the firmware or additional documentation about the RF protocol, further reverse engineering is next to impossible given the current \enquote{black box} state of the RF protocol. This means that one, unfortunately, cannot create new messages from scratch.

In conclusion, this pentest did not yield any additional actionable information. While the demodulated data definitely gives some insight into the system, it does not indicate any further vulnerabilities. Trying to reverse engineer the encryption method would yield a lot of additional information, and allow an attacker to create new messages from scratch. However, without access to the firmware, this is quite difficult and left as potential future work.

\input{chapters/pentesting-tasks/2-password-attack}
\input{chapters/pentesting-tasks/3-vesta-platform}
\section{Task 4: Insecure default credentials}
This section covers the topic of insecure default credentials. It does not include a \textit{pentest}, per se, but instead a list of all such credentials discovered in the system. These were discovered during the exploratory phase, and threat modeling phase of the project. Therefore, this section does not include a method part.

\subsubsection{Background}
Often default credentials are unavoidable. When first accessing the system, you need some way to do that without having authenticated yourself before. These credentials can, however, be more or less secure. A huge problem within cybersecurity is using insecure default credentials that can be easily guessed by a hacker. Additionally, often the user is never forced or even encouraged to change these credentials, leading to a potentially severe vulnerability. A common example is routers, which almost always feature an admin page to configure settings. Usually, all routers of the same model have the same default login credentials and often these are as simple as \texttt{admin:admin}. This is such a common issue that there are public databases of these passwords for each model \footnotelink{www.routerpasswords.com}{2021-0501}. Often the owner of the router is not even aware that this page exists or that they should change the password.

Famously, the MIRAI botnet relied on insecure default credentials to hack into hundreds of thousands of devices. It used just \texttt{62} different credentials \cite{understanding-mirai}. MIRAI is a worm malware, which targeted mostly IoT devices. It looked for devices with TCP port 23 or 2323 open, which are both often used by Telnet, and tried these credentials there. After gaining successful authentication against the device, these credentials and IP would be sent to a central server. Using this technique, over half a million devices were hacked and incorporated into the MIRAI botnet, which in turn was able to perform large scale DDOS attacks against websites. It was able to reach almost \texttt{1 TB/s} against a single target.

\subsubsection{Results}
The system can be armed and disarmed via the remote keypad (see section \ref{fig:hardware-components}) by entering a personal four-digit pin. By default, this is set to \texttt{1234}. You are never forced or even encouraged to enter a new code, overriding the easily-guessable default.

Additionally, when an alarm is triggered Securitas will send security personnel to inspect the site. If you accidentally triggered the alarm yourself Securitas has a phone number you can call to cancel the alarm. However, as a security mechanism, when doing this you have to correctly say a four-digit code (different from the previously discussed code). If you do not remember your code or say it incorrectly personnel will be sent to the site regardless. This code is by default the last four digits of \textit{your phone number}. Once again, you are never forced or even encouraged by the system to change this code.

No other bad default credentials were found. Importantly, the local admin panel (see section \ref{ch:system:software}) does not seem to use bad default credentials. Several hundred common default passwords were tried and none of them were successful (see section \ref{ch:pentesting:password}).

\subsubsection{Discussion}
Two insecure default credentials were discovered in the system. The first one, the arming pin defaulting to \texttt{1234}, is relatively severe. After triggering an alarm, you have a set time interval to disable it using your personal four-digit pin before the alarm is sent to the alarm center. An attacker could bet on the fact that this code has not been changed, and use it to turn off an alarm after breaking into the house. One could argue, however, that the probability of someone not changing this code is relatively low. The code arguably has an \textit{obvious} insecurity, leading to people perhaps feeling a greater need to change it. However, using a random code as the default would be substantially more secure.

The second code, used to call off an active alarm, is also a security issue. Your address and phone number can often easily be tied to one another and looked up using publicly available resources. An attacker obviously has the address if they are trying to get access to the house. Finding the phone number of the resident could be done in seconds using public websites for example. However, people can have multiple phone numbers and the house can of course have multiple residents.

Both identified insecure default credentials require the attacker to take quite a big risk to be able to exploit. An attacker would have to break into a house and only afterward bet on the fact that these had not been changed. One could then either turn off the alarm or try and call the alarm central before the owner does. All while the owner has been notified of an alarm breach via both email and the mobile app (see section \ref{ch:system:software}). While perhaps not critical vulnerabilities, they are nevertheless completely unnecessary. The system could easily give you a random code instead or force you to enter one upon first registering the system. Giving it an insecure default or tying it to publicly available information is not ideal and can be easily avoided.

