\chapter{Background} \label{ch:background}
% When you do your literature study, you should have a nearly complete Chapters 1 and 2. You may also find it convenient to introduce the future work section into your report early – so that you can put things that you think about but decide not to do now into this section. Note that later you can move things between this future work section and what you have done as you may change your mind about what to do now versus what to put off to future work.

% This chapter provides basic background information about xxx. Additionally, this chapter describes xxx. The chapter also describes related work xxxx.

% What does a reader (another x student -- where x is your study line) need to know to understand your report? What have others already done? (This is the “related work”.) Explain what and how prior work / prior research will be applied on or used in the degree project /work (described in this thesis). Explain why and what is not used in the degree project and give valid reasons for rejecting the work/research.

% Vilken viktig litteratur och (forsknings-)artiklar har du studerat inom området (litteraturstudie)?
[TODO]

\section{Major background area 1}
[TODO]

\subsection{Subarea 1.1}
[TODO]

\subsection{Subarea 1.1.2}
[TODO]

\subsection{Subarea 1.1.2}
[TODO]

\section{Major background area 2}
[TODO]

\section{Related work area}
[TODO]

\subsection{Major related work 1}
[TODO]

\subsection{Minor related work 1}
[TODO]

\section{Summary}
% Det är trevligt att få detta kapitel avslutas med en sammanfattning. Till exempel kan du inkludera en tabell som sammanfattar andras idéer och fördelar och nackdelar med varje - så som senare kan du jämföra din lösning till var och en av dessa. Detta kommer också att hjälpa dig att definiera de variabler som du kommer att använda för din utvärdering.

% It is nice to have this chapter conclude with a summary. For example, you can include a table that summarizes other people's ideas and benefits and drawbacks with each - so as later you can compare your solution to each of them. This will also help you define the variables that you will use for your evaluation.
[TODO]