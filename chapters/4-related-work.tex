\chapter{Related work} \label{ch:related-work}
This chapter described all identified related works done in the area and specifically against the hardware used in the system. Very little has been published regarding security of this system, as far as the author is aware. Two sources that explore systems based on the hardware from the same manufacturer (Climax Technology) have been identified. Additionally, a student thesis was done on a similar system but from a different manufacturer. These are presented below.

\section{Examination of LUPUS-Electronics devices} \label{ch:related-work:lupus}
This section details a security analysis of a very similar system, built on similar hardware from Climax Technology, done by \textit{Embedded Lab Vienna for IoT \& Security} (ELVIS)\footnotelink{https://www.elvis.science/}{2021-04-09}, a project at the University of Applied Sciences Campus Vienna in Austria.

Lupus Electronics is a German manufacturer of smart home security systems\footnotelink{https://www.lupus-electronics.de/en/}{2021-04-09}, much like \textit{Alarm.com} in the system examined in this thesis (see section \ref{ch:system:companies}). Just like \textit{Alarm.com}, \textit{Lupus Electronics} mostly provides the software of their system and also purchase hardware from the Taiwanese manufacturer \textit{Climax Technology}. In \citeyear{labvienna}, security researchers at ELVIS examined the security of the \textit{XT2 Plus Main Panel} from \textit{Lupus Electronics}. According to their report, similar hardware to the system examined in this thesis is used in that system. They found several vulnerabilities in the system. The most critical vulnerability they found was a Telnet server hosted on a strange high TCP port on the main panel. After examining the firmware of the system, and reverse engineering one of the applications, the researchers found that the password to the \texttt{root} user could be derived from a hardcoded salt and the MAC address of the panel. This meant that as long as an attacker had access to the local network, they could log in as root on the device, meaning with full privileges. With those privileges one could easily bypass the security of the alarm completely \citeyear{labvienna}.

While the Lupus system is not identical to the one in this thesis, much of the firmware from the hardware manufacturer is presumably the same. This assumption is strengthened by the fact that the endpoints found in the web interface of the Lupus system are the same in the thesis' system. The system under consideration in the report does not host a telnet server but does also have an application on a \textit{strange} high port.

According to their report, \textit{Climax Technology} was notified about the vulnerabilities on \texttt{2019-01-09} and a firmware revision fixing the vulnerabilities was released on \texttt{2019-03-26}.

\section{The Internet of Things: a privacy label for IoT products in a consumer market}
In their master thesis \citeauthor{iotprivacylabel} examines the design of an IoT privacy label, to help consumers recognize the security risks of the products they bring into their home. The main objective of their thesis is not strictly relevant to this report. However, the thesis includes three case studies of different consumer products, one of them being highly relevant to this project. In their thesis, they examine the security of a home alarm system from Egardia\footnotelink{https://www.egardia.com/en}{2021-04-12}, a dutch company producing smart home alarm systems. Much like \textit{Alarm.com} or \textit{Lupus Electronics} (see section \ref{ch:related-work:lupus}), Egardia are mainly responsible for the software platform and uses similar hardware from Climax Technology. While this system is not the main focus of the thesis, the author does report a few vulnerabilities. One is a replay attack vulnerability in the \gls{RF} communication.

\section{How Secure is Verisure’s Alarm System?}
\todo \cite{verisurethesis}
