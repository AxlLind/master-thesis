\chapter{Related work} \label{ch:related-work}
This chapter described all identified related works done in the area and specifically against the hardware used in the system. Very little has been published regarding security of this system, as far as the author is aware. Two sources that explore systems based on the hardware from the same manufacturer (Climax Technology) have been identified. Additionally, a student thesis was done on a similar system but from a different manufacturer. These are presented below.

\section{Related work 1: \textit{Examination of LUPUS-Electronics devices}} \label{ch:related-work:lupus}
This section details a security analysis of a very similar system, built on similar hardware from Climax Technology, done by \textit{Embedded Lab Vienna for IoT \& Security} (ELVIS)\footnotelink{https://www.elvis.science/}{2021-04-09}, a project at the University of Applied Sciences Campus Vienna in Austria.

Lupus Electronics is a German manufacturer of smart home security systems\footnotelink{https://www.lupus-electronics.de/en/}{2021-04-09}, much like \textit{Alarm.com} in the system examined in this thesis (see section \ref{ch:system:companies}). Just like \textit{Alarm.com}, \textit{Lupus Electronics} mostly provides the software of their system and also purchase hardware from the Taiwanese manufacturer \textit{Climax Technology}. In \citeyear{labvienna}, security researchers at ELVIS examined the security of the \textit{XT2 Plus Main Panel} from \textit{Lupus Electronics}. According to their report, similar hardware to the system examined in this thesis is used in that system. They found several vulnerabilities in the system. The most critical vulnerability they found was a Telnet server hosted on a strange high TCP port on the main panel. After examining the firmware of the system, and reverse engineering one of the applications, the researchers found that the password to the \texttt{root} user could be derived from a hardcoded salt and the MAC address of the panel. This meant that as long as an attacker had access to the local network, they could log in as root on the device, meaning with full privileges. With those privileges one could easily bypass the security of the alarm completely \citeyear{labvienna}.

While the Lupus system is not identical to the one in this thesis, much of the firmware from the hardware manufacturer is presumably the same. This assumption is strengthened by the fact that the endpoints found in the web interface of the Lupus system are the same in the thesis' system. The system under consideration in the report does not host a telnet server but does also have an application on a \textit{strange} high port. According to their report, \textit{Climax Technology} was notified about the vulnerabilities on \texttt{2019-01-09} and a firmware revision fixing the vulnerabilities was released on \texttt{2019-03-26}.

\section{Related work 2: \textit{The Internet of Things: a privacy label for IoT products in a consumer market}}
In their master thesis, author \citeauthor{iotprivacylabel} examines the design of an IoT privacy label, to help consumers recognize the security risks of the products they bring into their home. The main objective of their thesis, while interesting, is not strictly relevant to this report. However, the thesis includes three case studies of different consumer products, one of them being highly relevant to this project. They examine the security of a home alarm system from Egardia\footnotelink{https://www.egardia.com/en}{2021-04-12}, a dutch company producing smart home alarm systems. Much like \textit{Alarm.com} or \textit{Lupus Electronics} (see section \ref{ch:related-work:lupus}), Egardia are mainly responsible for the software platform and base their product on hardware from Climax Technology. While this system is not the main focus of their thesis, the author presents a short security analysis of the system, and reports a few vulnerabilities. One of them, and arguably the most critical, is a replay attack vulnerability in the \gls{RF} communication between the remote keypad and the central panel. The author demonstrates how using a \gls{SDR} to capture the \gls{RF} traffic of the user arming and disarming the system, can be simply replayed. The system is then armed/disarmed without issue. There seemed to be no mechanism in place to prevent this type of attack. While the author labeled this a medium security vulnerability, it has the potential to completely subvert the functionality of the system. All it requires is physical proximity to the system and a \gls{SDR}, which can be bought by anyone for about 300 USD\footnotelink{https://greatscottgadgets.com/hackrf/one/}{2021-04-12}. Additional vulnerabilities in the system were focused on the Egardia platform and is therefore not relevant to this thesis.

Like the system covered in section \ref{ch:related-work:lupus}, the Egardia system uses different hardware to the system in this thesis. However, they come from the same manufacturer, Climax Technology, and they both communicate on the same radio frequency (868 MHz). There is strong reason to believe that they use the same underlying \gls{RF} protocol, a proprietary one from the hardware manufacturer called F1\footnotelink{https://www.climax.com.tw/new/f1-features-new.php}{2021-04-12}. This could mean that the system covered in this thesis is vulnerable to similar attacks on the \gls{RF} protocol.

\section{Related work 3: \textit{How Secure is Verisure’s Alarm System?}}
In their thesis, authors \citeauthor{verisurethesis} examine the cyber security of a home alarm system from \textit{Verisure}. The company claims to sell the most widely installed home alarm in Europe\footnotelink{https://www.verisure.se/english}{2021-04-12}, installed in over 350 000 homes in Sweden alone. The thesis mostly focus on the SCTP communication between the main panel and the external Verisure servers, as well as the web security of the Verisure software platform. The authors found several \gls{CSRF} vulnerabilities, allowing an attacker to disarm the system and create new users.

The examined system from Versiure, from the point of view of a user, is similar. It offers almost the same components and features. From a technical perspective, however, the two systems are quite different. There is no overlap in the hardware components and much of the network technologies used are different. The main panel in the Verisure system for example communicated over a broadband connection using the SCTP protocol, in contrast to the system in this thesis which uses 3G telecommunication. Their work does, however, show that the cyber security of similar products in the industry may be lacking.
