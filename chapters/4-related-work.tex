\chapter{Related work} \label{ch:related-work}
This chapter describes all identified related works done in the area and specifically against the hardware used in the system. Very little has been published regarding the cybersecurity of this system, as far as the author is aware. Two sources that explore systems based on the hardware from the same manufacturer, Climax Technology, have been identified. Additionally, a student thesis was done on a similar system but from a different manufacturer. These are presented below.

\section{OWASP IoT Top 10}
The OWASP top ten for web vulnerabilities is one of the most used sources of the most common web vulnerabilities \cite{owasp-www-top10}. The OWASP Foundation, the creators of the list, even calls it \textit{"Globally recognized by developers as the first step towards more secure coding"}\footnotelink{https://owasp.org/www-project-top-ten/}{2021-05-21}. OWASP also publishes a list of the top ten most common vulnerabilities for IoT systems \cite{owasp-iot-top10}. The latest revision was released in \citeyear{owasp-iot-top10}. This list is of highly relevant to this report, as the system in question is an IoT system. The top ten vulnerabilities for IoT systems, according to OWASP's report are the following, in order of importance:
\begin{enumerate}
    \item \textbf{Weak, Guessable, or Hardcoded Passwords}. This includes passwords that can be easily bruteforced, are publicly. available, or unchangeable. Famously the MIRAI botnet, which mostly included IoT devices, managed to hack over half a million devices by testing just 60 default credentials \cite{understanding-mirai}.
    
    \item \textbf{Insecure Network Services}. Often unneeded network services will be left open on IoT devices, even when in production. Commonly, these are used during development but are never removed when the device is shipped. An example is open telnet services, or ssh, being left open, leaving a backdoor open for the manufacturer. However, these network services also open the device up to a whole slew of potential vulnerabilities and are often completely unnecessary for the functionality of the IoT system. A real world example of this is again the MIRAI botnet, which scanned IP ranges for devices with telnet services hosted on port 21 or 2121 \cite{understanding-mirai}.
    
    \item \textbf{Insecure Ecosystem Interfaces}. Many IoT devices are much more than the device itself. There is often an entire infrastructure behind it to control and interact with the system, such as mobile application or a website, with an accompanying backend API server. Vulnerabilities in these systems can lead to the IoT device itself being compromised.
    
    \item \textbf{Lack of Secure Update Mechanism}. To add new features and to fix security problems, updating the firmware of IoT devices is unavoidable. However, over-the-air (OTA) firmware updates introduces an additional attack vector and can have many potential security issues. This can include the lack of firmware validation on the device, delivering the firmware unencrypted in transit, lack of a mechanism to prevent rolling back changes, and a lack of notifications of security changes due to updates.
    
    \item \textbf{Use of Insecure or Outdated Components}. This means using old versions of libraries and components that themselves are vulnerable, which in turn can compromise the security of the system as a whole. This can include everything from insecure versions of operating systems, insecure customization of the OS, third-party software, and hardware components from higher up in the supply chain.
    
    \item \textbf{Insufficient Privacy Protection}. This refers to the user's personal information being stored in an insecure way. Either on the IoT system itself or in the larger ecosystem.
    
    \item \textbf{Insecure Data Transfer and Storage}. Lack of encryption or proper access control, both in transit and in storage. This can, once again, be both on the IoT device and anywhere else in the larger ecosystem.
    
    \item \textbf{Lack of Device Management}. This includes lack of secure asset management, update management, secure decommissioning of the system, proper monitoring of the system, as well as capabilities to respond to a security issue.
    
    \item \textbf{Insecure Default Settings}. This refers to the device being sold with insecure default settings. Often the user is never prompted or encouraged to change these insecure defaults.
    
    \item \textbf{Lack of Physical Hardening}. During development, debug hardware interfaces are used. However, a common vulnerability is not removing these before shipping the product. This can include open JTAG or UART interfaces on the circuit board, allowing for extraction of the firmware or even terminal access to the system over a serial interface. Often these vulnerabilities require physical access to the system. However, they can be used to extract valuable information which can help an attacker in developing remote exploits.
\end{enumerate}

\section{Related work 1: \textit{Examination of LUPUS-Electronics devices}} \label{ch:related-work:lupus}
This section details a security analysis of a very similar system, built on similar hardware from Climax Technology. The study was conducted by members of \textit{Embedded Lab Vienna for IoT \& Security} (ELVIS)\footnotelink{https://www.elvis.science/}{2021-04-09}, a project at the University of Applied Sciences Campus Vienna in Austria.

\textit{Lupus Electronics} is a German manufacturer of smart home security systems\footnotelink{https://www.lupus-electronics.de/en/}{2021-04-09}, much like \textit{Alarm.com} in the system examined in this thesis (see section \ref{ch:system:companies}). Just like Alarm.com, Lupus Electronics mostly provides the software of their system and also purchase hardware from the Taiwanese manufacturer \textit{Climax Technology}. In \citeyear{labvienna}, security researchers at ELVIS examined the security of the \textit{XT2 Plus Main Panel} from Lupus Electronics. According to their report, similar hardware to the system examined in this thesis is used in that system. They reported several vulnerabilities. The most critical vulnerability they found was a Telnet server hosted on a non-standard high TCP port on the main panel. After examining the firmware of the system, and reverse engineering one of the applications, the researchers found that the password to the \texttt{root} user could be derived from a hardcoded salt and the MAC address of the panel. This meant that as long as an attacker had access to the local network, they could log in as root on the device, meaning with full privileges. With those privileges, one could easily bypass the security of the alarm completely.

While the Lupus system is not identical to the one in this thesis, much of the firmware from the hardware manufacturer is presumably the same. This assumption is strengthened by the fact that the endpoints found in the web interface of the Lupus system are the same in the thesis' system. The system under consideration in this report does not host a telnet server, however, it does host an application listening on a \textit{non-standard} high TCP port. According to their report, Climax Technology was notified about the vulnerabilities on \texttt{2019-01-09} and a firmware revision fixing the vulnerabilities was released on \texttt{2019-03-26}.

\section{Related work 2: \textit{The Internet of Things: a privacy label for IoT products in a consumer market}}
In their master thesis, author \citeauthor{iotprivacylabel} examines the design of an IoT privacy label, to help consumers recognize the security risks of the products they bring into their home \cite{iotprivacylabel}. The main objective of their thesis, while interesting, is not strictly relevant to this report. However, the thesis includes three case studies of different consumer products, one of them being highly relevant to this project. They examine the security of a home alarm system from \textit{Egardia}\footnotelink{https://www.egardia.com}{2021-04-12}, a dutch company producing smart home alarm systems. Much like \textit{Alarm.com} or \textit{Lupus Electronics} (see section \ref{ch:related-work:lupus}), Egardia are mainly responsible for the software platform and base their product on hardware from Climax Technology. While this system is not the main focus of their thesis, the author presents a short security analysis of the system and reports a few vulnerabilities. One of them, and arguably the most critical, is a replay attack vulnerability in the \gls{RF} communication between the remote keypad and the central panel. The author demonstrates how using a \gls{SDR} to capture the \gls{RF} traffic of the user arming and disarming the system, can be simply replayed. The system is then armed/disarmed without issue. There was to be no mechanism in place to prevent this type of attack. While the author labeled this a medium-level security vulnerability, it has the potential to completely subvert the functionality of the system. All it requires is physical proximity to the system and a \gls{SDR}, which can be bought by anyone for about 350 USD\footnotelink{https://greatscottgadgets.com/hackrf/one/}{2021-04-12}. Additional vulnerabilities in the system were related to the Egardia software platform and are therefore not relevant to this thesis.

Like the system covered in section \ref{ch:related-work:lupus}, the Egardia system uses different hardware to the system in this thesis. However, they come from the same manufacturer, Climax Technology, and they both communicate on the same radio frequency (868 MHz). There is strong reason to believe that they use the same underlying \gls{RF} protocol, a proprietary one from the hardware manufacturer called F1\footnotelink{https://www.climax.com.tw/new/f1-features-new.php}{2021-04-12}. This could mean that the system covered in this thesis is vulnerable to similar attacks on the \gls{RF} protocol.

\section{Related work 3: \textit{How Secure is Verisure’s Alarm System?}}
In their thesis, authors \citeauthor{verisurethesis} examine the cybersecurity of a home alarm system from \textit{Verisure}. The company claims to sell the most widely installed home alarm in Europe\footnotelink{https://www.verisure.se/english}{2021-04-12}, installed in over 350 000 homes in Sweden alone. Their thesis mostly focuses on the SCTP communication between the main panel and the external Verisure servers, as well as the web security of the Verisure software platform. The authors found several \gls{CSRF} vulnerabilities, allowing an attacker to disarm the system and create new users.

The examined system from Versiure, from the point of view of a user, is similar to the one in this report. It offers almost the same components and features. From a technical perspective, however, the two systems are quite different. There is no overlap in the hardware components and much of the network technologies used are different. For example, the main panel in the Verisure system communicates over a broadband connection using the SCTP protocol, in contrast to the system in this thesis that uses 3G telecommunication. Their work does, however, show that the cybersecurity of similar products in the industry may be lacking.
