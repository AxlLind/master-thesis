\section{Task 7: Insecure default credentials}
This section covers the topic of insecure default credentials. It does not include a \textit{pentest}, per se, but instead a list of all such credentials discovered in the system. These were discovered during the exploratory phase, and threat modeling phase of the project. Therefore, this section does not include a method part.

\subsubsection{Background}
Often default credentials are unavoidable. When first accessing the system, you need some way to do that without having authenticated yourself before. These credentials can, however, be more or less secure. A huge problem within cybersecurity is using insecure default credentials that can be easily guessed by a hacker. Additionally, often the user is never forced or even encouraged to change these credentials, leading to a potentially severe vulnerability. A common example is routers, which almost always feature an admin page to configure settings. Usually, all routers of the same model have the same default login credentials and often these are as simple as \texttt{admin:admin}. This is such a common issue that there are public databases of these passwords for each model \footnotelink{www.routerpasswords.com}{2021-05-01}. Often the owner of the router is not even aware that this page exists or that they should change the password. OWASP ranks this as the number one most common and severe vulnerability in IoT systems \cite{owasp-iot-top10}.

Famously, the MIRAI botnet relied on insecure default credentials to hack into hundreds of thousands of devices. It used just \texttt{62} different credentials \cite{understanding-mirai}. MIRAI is a worm malware, which targeted mostly IoT devices. It looked for devices with TCP port 23 or 2323 open, which are both often used by Telnet, and tried these credentials there. After gaining successful authentication against the device, these credentials and IP would be sent to a central server. Using this technique, over half a million devices were hacked and incorporated into the MIRAI botnet, which in turn was able to perform large scale DDOS attacks against websites. It was able to reach almost \texttt{1 TB/s} against a single target.

\subsubsection{Results}
The system can be armed and disarmed via the remote keypad (see section \ref{fig:hardware-components}) by entering a personal four-digit pin. By default, this is set to \texttt{1234}. You are never forced or even encouraged to enter a new code, overriding the easily-guessable default.

Additionally, when an alarm is triggered Securitas will send security personnel to inspect the site. If you accidentally triggered the alarm yourself Securitas has a phone number you can call to cancel the alarm. However, as a security mechanism, when doing this you have to correctly say a four-digit code (different from the previously discussed code). If you do not remember your code or say it incorrectly personnel will be sent to the site regardless. This code is by default the last four digits of \textit{your phone number}. Once again, you are never forced or even encouraged by the system to change this code.

No other bad default credentials were found. Importantly, the local admin panel (see section \ref{ch:system:local-admin}) does not seem to use bad default credentials. Several hundred common default passwords were tried and none of them were successful (see section \ref{ch:pentesting:password}).

\subsubsection{Discussion}
Two insecure default credentials were discovered in the system. The first one, the arming pin defaulting to \texttt{1234}, is relatively severe. After triggering an alarm, you have a set time interval to disable it using your personal four-digit pin before the alarm is sent to the alarm center. An attacker could bet on the fact that this code has not been changed, and use it to turn off an alarm after breaking into the house. One could argue, however, that the probability of someone not changing this code is relatively low. The code arguably has an \textit{obvious} insecurity, leading to people perhaps feeling a greater need to change it. However, using a random code as the default would be substantially more secure.

The second code, used to call off an active alarm, is also a security issue. Your address and phone number can often easily be tied to one another and looked up using publicly available resources. An attacker obviously has the address if they are trying to get access to the house. Finding the phone number of the resident could be done in seconds using public websites for example. However, people can have multiple phone numbers and the house can of course have multiple residents.

Both identified insecure default credentials require the attacker to take quite a big risk to be able to exploit. An attacker would have to break into a house and only afterward bet on the fact that these had not been changed. One could then either turn off the alarm or try and call the alarm central before the owner does. All while the owner has been notified of an alarm breach via both email and the mobile app (see section \ref{ch:system:mobile-app}). While perhaps not critical vulnerabilities, they are nevertheless completely unnecessary. The system could easily give you a random code instead or force you to enter one upon first registering the system. Giving it an insecure default or tying it to publicly available information is not ideal and can be easily avoided.
