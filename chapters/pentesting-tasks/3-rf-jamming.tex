\section{Task 3: RF Jamming Attack} \label{ch:pentesting:rf-jamming}
This section covers a pentest of a jamming attack against the RF communication in the system. This is a \gls{DOS} attack which if successful blocks or interferes with the RF communication between two devices, making them unable to send messages to each other.

\subsubsection{Background}
Many wireless communication mediums are vulnerable to jamming attacks. Radio frequency communication is certainly one of them. A jamming attack against RF communication involves directing electromagnetic energy in one or more radio frequencies against a system to disrupt or prevent signals from being transmitted between two systems \cite{adamy2004ew}. In practice, this means sending out signals on a specific frequency, carrying enough energy to overpower any one transmission in the same frequency band. By continuously sending out signals, such that the wireless band is filled, legitimate traffic can be blocked. Since RF communication uses a shared medium, this is an attack vector which can be incredibly hard to protect against. Often a system will communicate on a single, fixed frequency which can make the system particularly vulnerable to jamming attacks \cite{jamming-feasibility}.

\subsubsection{Method}
To transmit signals the HackRF SDR was used. It was placed close to the system, within \texttt{10-20 cm}. See section \ref{ch:pentesting:lab-setup} for a detailed description of the lab setup. To generate a jamming signal the open source program \textit{GnuRadio Companion}\footnotelink{https://www.gnuradio.org/}{2021-05-22}, version \texttt{3.9.0.0}, was used. This is a graphical tool used to control an SDR. It is based on creating flowgraphs of connected components to receive, process, modify, and transmit real time radio signals from and to an SDR.
\begin{figure}[!ht]
    \centering
    \includegraphics[width=\textwidth]{images/6-pentesting/jamming-flowgraph.png}
    \caption{A flowgraph in GnuRadio which performs a jamming attack.}
    \label{fig:gnuradio-jamming-flowgraph}
\end{figure}
To generate a noise signal, a flowgraph was created in GnuRadio Companion, see figure \ref{fig:gnuradio-jamming-flowgraph}. Initially, a \texttt{Fast Noise Generator} was used as the source signal. The output was then linked to a low-pass filter, to concentrate the signals to the specific frequency band of interest. Lastly, the output was sent to the HackRF via the \texttt{osmocom Sink} block. Additionally, the output from the low-pass filter was also sent to a \texttt{QT GUI Frequency Sink} to visually present the sent signal data while performing the attack. This is shown in figure \ref{fig:gnuradio-frequency-graph}.
\begin{figure}[!ht]
    \centering
    \includegraphics[width=\textwidth]{images/6-pentesting/jamming-output-graph.png}
    \caption{A frequency graph from GnuRadio during the jamming attack.}
    \label{fig:gnuradio-frequency-graph}
\end{figure}

\subsubsection{Results}
This pentest was successful. Signals between the devices are jamming. However, after approximately 60 seconds of running the attack the main panel registers an \textit{interference fault event}. While this fault event is registered and the jamming attack is successfully detected by the system, crucially the event is not logged in the mobile or web application and no notification is sent to the owner of the system what so ever.

\subsubsection{Discussion}
As expected, the system fails to communicate during the jamming attack. However, appropriate measures have arguably been taken by the manufacturer. Given the inherit vulnerability of jamming attacks in a shared medium like this, detecting and reporting is the only reasonable course of action. This behavior documented in the official user manual submitted to the FCC\cite{hsgw-user-manual} by Climax Technology. It describes in the \textit{interference} status in section 6.1 that if a jamming period of 30 seconds or more during a one minute window is detected then the event is triggered. This lines up with the behavior examined during the test. However, the system fails to log and report the attack to the user.