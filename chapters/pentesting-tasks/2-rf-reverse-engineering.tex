\section{Task 2: Reverse engineering the RF protocol} \label{ch:pentesting:rf-reverse-engineering}
This section covers the process of reverse engineering the RF protocol, via captured RF signals. This included mainly trying to demodulate the signals into binary data and then analyzing the contents.

\subsection{Background}
The system in question uses RF signals to wirelessly communicate across the devices. These radio waves are used to transfer binary data, ones and zeroes, between each device. Transferring binary data over radio waves is done by a process called modulation \cite{rf-modulation}. Digital modulation, e.g modulating binary data, is a whole field of study on to it self and that this is only a brief overview covering the basics of modulation.

There are three primary simple ways of modulating a binary signal: Amplitude- (ASK), Frequency- (FSK), and Phase- (PSK) shift keying. They each produce a distinct wave form, which can easily be visually identified. This is shown in figure \ref{fig:digital-modulation}.
\begin{figure}[!ht]
    % Perhaps I don't have the license to use this?
    \centering
    \includegraphics[width=0.7\textwidth]{images/6-pentesting/digital-modulation.png}
    \caption{The three primary techniques for digital modulation.}
    \label{fig:digital-modulation}
\end{figure}
Each technique uses a different property of the sinusoidal wave to encode a zero and one respectively. \gls{ASK} uses the amplitude of the wave, where a higher amplitude usually encodes a \texttt{1} and a lower amplitude a \texttt{0}. The frequency and phase of the wave is kept constant. \gls{FSK}, on the other hand, keeps both amplitude and phase constant. Instead two different frequencies are shifted between to differentiate between a \texttt{0} or \texttt{1}. Lastly, \gls{PSK} uses a phase-shift to differentiate between the two symbols. There are many much more complicated techniques to more efficiently encode a binary signal in radio waves \cite{rf-modulation}. However, these are not covered in this report.

Conversely, \textit{demodulation} is the process of covering a signal back into binary data. In most applications this is done usually done directly in hardware, using specific receivers and circuitry to automatically convert the signal back into binary data \cite{rf-modulation}. Often this hardware also implements error correction and other techniques to increase reliability of the communication, as interference and other disturbances are unavoidable in the real world. Demodulation can of course also be done in software, but doing so in real-time is quite CPU-intensive and consequently a lot less energy efficient.

\subsection{Method}
Initially, signals were captured according to the method described in section \ref{ch:pentesting:replay:method}. Carefully inspecting the signals in \gls{URH}, as shown in figure \ref{fig:rf-signal-capture}, one can see that clearly \gls{FSK} modulation is used to modulate the signal. This conclusion is strengthened by official test documentation submitted to the \textit{Telecommunications Bureau of the Ministry of Internal Affairs and Communications} (MIC) in Japan \footnotelink{https://www.tele.soumu.go.jp/giteki/SearchServlet2?PageID=jt01&ATF=12866003}{2021-05-17}, in which the modulation type is specified to be FSK. This organization is similar to the FCC in the US. However, the information submitted to the FCC does not mention modulation type. This test document submitted to MIC was found through some extensive \gls{OSINT} and Google.

Furthermore, \gls{URH} boasts that one of their key features is automatic demodulation \cite{urh}. This was tested on all captured signals. However, it was unfortunately continuously unsuccessful to automatically find the correct parameters of the modulation. Instead, the signals were demodulated \textit{by hand} using the open source audio processing tool \textit{Audacity}\footnotelink{https://www.audacityteam.org/}{2021-05-17}. This was done in the following steps:
\begin{enumerate}
    \item The raw signal data, as captured by URH, was imported into Audacity via the menu options \texttt{File -> Import -> Raw Data}. URH saves the signal as raw IQ data of signed bytes. By selecting the encoding \texttt{Signed 8-bit PCM} with \texttt{2 Channels (Stereo)}, the I and Q components of the data gets imported into two separate tracks. We only care about one of them in this case and as such the stereo track was split into two mono tracks in Audacity and the second one was then deleted. The rest of the import options do not matter.
    \item Secondly, the center frequency, as perceived by Audacity, was noted. This can be found by the menu options \texttt{Analyze -> Plot Spectrum}. The plot is shown in figure \ref{fig:audacity-spectrum}. Note that due to several features like the sample rate not being part of the raw signal data, this frequency might not equal the actual center frequency of \texttt{868.64 MHz}. This frequency was used in the steps below.
    \item A \texttt{High-Pass Filter} was applied to the first track, and a \texttt{Low-Pass Filter} to the second. The former essentially maps a higher frequency to a higher amplitude in the output wave, and the latter does the opposite. Then the amplitude of both tracks was slightly increased.
\end{enumerate}
\begin{figure}[!ht]
    \centering
    \includegraphics[width=0.7\textwidth]{images/6-pentesting/audacity-spectrum.png}
    \caption{A spectrum plot in Audacity, used to find the center frequency.}
    \label{fig:audacity-spectrum}
\end{figure}

\subsection{Results}
\todo

\subsection{Discussion}
\todo
