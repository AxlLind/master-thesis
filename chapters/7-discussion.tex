\chapter{Discussion} \label{ch:discussion}
This chapter contains a discussion about the methodology used in this report, as well as a discussion of the results found in chapter \ref{ch:pentesting}. Lastly, a mandated discussion about the sustainability and ethics of this project is presented.

\section{Methodology}
The methodology used in this report is described extensively in chapter \ref{ch:method}. Largely, the methodology applied in this thesis was effective. It gave a clear structure to follow and the threat model gave insights into what type of vulnerabilities to consider. If one were to blindly guess and pentest the system without any structure it likely one could miss some obvious threats. By following established attack libraries like the OWASP IoT Top 10 \cite{owasp-iot-top10} (see section \ref{ch:related-work:owasp}) and the ETSI EN 303 645 standard, it was much easier to identify reasonable vulnerabilities to explore. Otherwise, one could quite easily get caught in dead ends with low chances of success.

A seven step penetration testing methodology described by Weidman \cite{weidman2014} was applied in this thesis. It was intuitive and easy to follow. However, one downside was that it perhaps was a bit excessive for this type of project. Being time constraint was a constant theme during this thesis. Things had to be continuously delimited as the project went on and the deadline approached. Steps like ranking each threat using the DREAD model, for example, was suggested in the methodology but was decided to not be worth it for such a time limited project. Since there were more promising threats to examine than there was time, a more intuitive approach was used when deciding on what threats to focus on. Another downside was the inexperience of the author when it comes to threat modeling and pentesting methodology. It is not something that had been covered previously in the education at KTH. Therefore, a lot of the initial time of the thesis was spent on researching this area.

Overall, however, the methodology was very effective and gave good results. It gave structure to the penetration testing phase, which is otherwise often driven by intuition and probing the system looking for common sources of vulnerabilities. This is often the case during ethical hacking.

\section{Results}
The results from the penetration testing shows that the system, in its current firmware version, unfortunately cannot be considered secure. It is clear that some considerable thought has gone into the security of the system by the manufacturer. The system features tamper sensors to protect against physical attacks, it has a backup battery to guard against power outages, and it uses 3G telecommunication so as to not rely on the connectivity of the local network. Even on the RF level some attention has been given to security. The system is able to detect RF jamming, and implements some kind of encryption or obfuscation. However, none of that matters. When some thing as trivial as a replay attack is not protected against, and has such critical consequences, it makes all of the other security measures irrelevant. They are completely negated.

These results are made even more severe considering that the critical vulnerability is what is known as a \enquote{zero knowledge} attack. It requires no knowledge about the system and RF protocol from the attacker. Anyone with 350 USD left over to buy an SDR, and some very basic technical competence could perform the attack. Tools like \textit{Universal Radio Hacker} \cite{urh} presents this process visually through an easy to use GUI, making it approachable to almost anyone. The reassurance, however, is that it requires capturing a live signal, e.g. someone actually leaving the house and arming the system. This requires some dedication from the attackers, waiting potentially a long time until the right moment. It also requires physical proximity, however, that is already a given if one wants to enter a property.

Additionally, the results revealed several other promising avenues to explore in the system. For example, it hosts a suspicious network service, on \textit{58098/tcp}. This is quite worrying, as insecure network services is one of the most common vulnerabilities against IoT systems \cite{owasp-iot-top10}. While this application could not be reverse-engineered in this project, one might be able to if the firmware of the system was acquired. There is potential that this could cause additional severe vulnerabilities. Additionally, one could potentially reverse engineer the RF protocol by analyzing the firmware, and be able to construct and transmit arbitrary messages. More on this in section \ref{ch:conclusion:related-work}.

The results are reliable. Each penetration test was grounded in an extensive background section, and they were all successfully repeated multiple times.

\section{Sustainability and Ethics}
In the field of computer security, ethics is always present. The findings of penetration testing, or hacking, could have very big consequences and could in some cases be incredibly damaging to the target. This makes ethical concerns very present when performing penetration testing and security analysis. One has to take great care when hacking this so as to not harm or disrupt. When a vulnerability is discovered another ethical concern comes forth, namely how to disclose it. [...] \todo (unfinished)
