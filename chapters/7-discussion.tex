\chapter{Discussion} \label{ch:discussion}
This chapter contains a discussion about the methodology used in this report, as well as a discussion of the results found in chapter \ref{ch:pentesting}. Lastly, a mandated discussion about the sustainability and ethics of this project is presented.

\section{Methodology}
The methodology used in this report is described extensively in chapter \ref{ch:method}. Largely, it was effective. It gave a clear structure to follow and the threat model gave insights into what type of vulnerabilities to consider. If one were to blindly guess and pentest the system without any structure it would have been perhaps unavoidable to not miss some obvious threats. By following established attack libraries like the OWASP IoT Top 10 \cite{owasp-iot-top10} (see section \ref{ch:related-work:owasp}) and the ETSI EN 303 645 standard, it was much easier to identify reasonable vulnerabilities to explore. It would have otherwise been quite easy to get caught in dead ends with low chances of success.

The seven step penetration testing methodology described by Weidman felt intuitive and easy to follow. However, one downside was that it was perhaps a bit overkill for this type of project. This thesis always felt quite time constrained. Things had to be continuously delimited as the project went on and the deadline approached. Steps like ranking each threat using the DREAD model, for example, was suggested in the methodology but was decided to be not worth it for such a time constrained project. Since there were more promising threats to examine than there was time, a more intuitive approach was made when deciding on what threats to focus on. Another downside was the inexperience of the author when it comes to threat modeling and pentesting methodology. It is not something that had been covered at all previously in the education at KTH. This meant a lot of the initial time of the thesis was spent on researching this area.

However, overall the methodology was very effective and gave good results. It gave structure to the penetration testing, which is otherwise often driven by intuition, simply looking around, and \enquote{randomly} probing the system looking for common vulnerabilities, as is often the case during ethical hacking.

\section{Results}
The results from the penetration testing shows that, unfortunately, the system, in its current firmware version, cannot be considered secure. It is clear that some considerable thought has gone into the security of the system. The system features tamper sensors to protect against physical attacks, it has a backup battery to guard against power outages, and it uses 3G telecommunication so as to not rely on the connectivity of the local network. Even on the RF level some attention has been given to security. The system is able to detect RF jamming, and implements some kind of encryption or obfuscation. However, none of that matters. When some thing as trivial as a replay attack is not protected against, and has such critical consequences, none of the other security measures matters. They are completely negated.

These results are made even more severe considering that the critical vulnerability is what is known as a \enquote{zero knowledge} attack. It requires no knowledge about the system and RF protocol from the attacker. Any one with 350 USD over to buy an SDR, and some very basic technical competence could perform the attack. Tools like \textit{Universal Radio Hacker} \cite{urh} presents this process visually through an easy to use GUI, making it approachable to almost any one. The reassurance, however, is that it requires capturing a live signal, e.g someone actually leaving the house and arming the system. This takes some dedication from the attackers, waiting potentially a long time until the right moment. It also requires physical proximity, however, that is already a given if one wants to enter a property.

Additionally, the results revealed several other promising avenues to explore in the system. For example, it hosts a suspicious network service, on \textit{58098/tcp}. This is quite worrying, as insecure network services is one of the most common vulnerabilities against IoT systems \cite{owasp-iot-top10}. While this application could not be reverse-engineered in this project, if one were to acquire the firmware of the system there is potential that this could be a severe vulnerability. Additionally, one could potentially reverse engineer the RF protocol, and be able to construct and transmit arbitrary messages. More in this in section \ref{ch:conclusion:related-work}.

The results are reliable. Each penetration test was grounded in an extensive background section, and they were all successfully repeated several times.

\section{Sustainability and Ethics}
\todo
