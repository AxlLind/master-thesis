\chapter{Conclusion \& Future Work} \label{ch:conclusion}
This final chapter contains the conclusions drawn from the result of this thesis and a conclusion about the systems overall security. Additionally, a discussion about future work that could be done on the security of the examined system is presented.

\section{Conclusion}
Is the SecuritasHome Home Alarm System secure against cyberattacks? The answer to the research question has to be \textbf{no}. It is clear that the manufacturer has put some considerable thought into security. They use tamper sensors on all devices to protect against physical attacks, it has a battery to protect against a power outage, they use 3G telecommunication so as to not rely on the local network connection, they are able to detect jamming attacks, and they use some kind of encryption in the RF protocol (the cryptographic security of which is still in question). However, due to a glaring security flaw in the RF protocol, not protecting against replay attacks, these measures are made largely irrelevant. It goes to show how one mistake is all it takes to completely negate the security of an IT system.

Additionally, there are some bad practices found in the system, in clear violation of the ETSI EN 303 645 standard for IoT manufactures \cite{etsi-iot-standard}. One of them is leaving several network services on the system. They seemingly have no bearing on the functionality of the system and only serve to increase its attack surface, which is cause to worry.

In conclusion, the system is vulnerable to RF replay attacks. This lets an attacker disarm the system. While many security measures have been taken, this completely negates its security.

\section{Future work} \label{ch:conclusion:related-work}
There is a lot left to examine regarding the security of the SecuritasHome smart alarm system. This project focused on the RF protocol as well as the systems network services. The results, however, opened up many additional promising avenues to explore. Due to time constraints these could not be included in this thesis and several interesting aspects had to be delimited.

Firstly, somehow acquiring the firmware of the system would open the door for a lot of interesting research. Without it, analyzing the behavior and purpose of the \textit{58098/tcp} network service for example is very difficult. That would probably be possible to ascertain by reverse engineering the firmware. One possible technique to acquire the firmware, since it is not published on the manufacturers website, would be to solder off the flash memory from the PCB and read its contents. However, that is a quite risky process that would certainly permanently break the system and possibly the flash memory in the process. The system supports over-the-air firmware updates, however, no firmware update was recorded during this entire project, lasting about 6 months. Catching an OTA firmware update in the act and thereby acquiring the firmware is therefore quite difficult.

Additionally, one could possibly reverse engineer the encryption method used in the RF protocol by reverse engineering the firmware. An interesting avenue is using publicly available firmware from a similar system to do this. The alarm system sold by \textit{Lupus Electronics}, as discussed in section \ref{ch:related-work:lupus}, for some reason publish the firmware openly on their website\footnotelink{https://www.lupus-electronics.de/en/service/downloads/}{2021-06-09}. Their system is also from Climax Technology and supports F1 compatible products, the proprietary RF protocol. It is however using a different panel model. Analyzing that firmware, it is quite probable one could reverse engineer the encryption scheme used, as well as figure out the message structure used in the protocol. However, there was not enough time to include that analysis in this thesis.

The surrounding ecosystem, including the website and mobile application is another interesting area to explore. This was delimited early on due to both the difficulty regarding the legal aspects, and due to the interests of the author. Analyzing the android app, for example, could reveal interesting exploits. This is often quite approachable since one can decompile APKs to something quite close to the original source code. Analyzing the API used by both the website and mobile app is another interesting area.

% And we're done
% ( •_•)
% ( •_•)>⌐■-■
% (⌐■_■)
\noindent\rule{\textwidth}{0.4mm}
