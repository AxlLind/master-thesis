\titlepage
\bookinfopage

\frontmatter

\begin{abstract}
\markboth{\abstractname}{}
The number of IoT systems in our homes has exploded in recent years. By 2025 it is expected that the number of IoT devices will reach 38 billion. Home alarm systems are an IoT product that has increased dramatically in number and complexity in recent years. Besides triggering an alarm when an intruder tries to break in, a modern system can now control your light bulbs, lock and unlock your front door remotely, and interact with your smart speaker. They are undeniably effective in deterring physical intrusion. However, given the recent rise in complexity how well do they hold up against cyber attacks?

In this thesis, a smart home alarm system from SecuritasHome is examined. A comprehensive security analysis was performed using penetration testing techniques and threat modeling. The work focused mainly on radio frequency (RF) hacking against the systems RF communication. Among other things, a critical vulnerability was found in the proprietary RF protocol, allowing an attacker to disarm an armed system and thus completely bypass the system's functionality. The security of the system was deemed to be lacking.

\subsection*{Keywords}
penetration testing, threat modeling, IoT, computer security, home alarm system

\end{abstract}

\clearpage

\selectlanguage{swedish}
\begin{abstract}
Antalet IoT system i våra hem har exploderat de senaste åren. Vid år 2025 förväntas antalet IoT enheter nå 38 miljarder. Hemlarmsystem är en typ av IoT-produkt som ökat dramatiskt i komplexitet på senare tid. Förutom att framkalla ett larm vid ett intrång kan ett modernt hemlamsystem numera kontrollera dina glödlampor, låsa och låsa upp din ytterdörr, samt kontrollera dina övervakningskameror. De är utan tvekan effektiva på att förhindra fysiska intrång, men hur väl står de emot cyberattacker?

I denna uppsats undersöks ett hemlarmsystem från SecuritasHome. En utförlig säkerhetsanalys gjordes av systemet med penetrationstestnings-metodiker och hotmodellering. Arbetet fokuserade mestadels på radiovågshackning (RF) mot systemets RF-kommunikation. Bland annat hittades en kritiskt sårbarhet i systemets RF-protokoll som gör det möjligt för en angripare att avlarma ett larmat system, och därmed kringå hela systemets funktionalitet. Säkerheten av systemet bedömdes vara bristfällig.

\subsection*{Nyckelord}
penetrationstestning,hotmodellering, IoT, datasäkerhet, hemlarmsystem

\end{abstract}

\clearpage

\selectlanguage{english}
\section*{Acknowledgments}
\markboth{Acknowledgments}{}
I would like to thank Fredrik Heiding, PhD student within cybersecurity at KTH, for helping me procure the alarm system investigated in this thesis, as well as the HackRF SDR, and navigating the KTH bureaucracy surrounding that process.

I would also like to acknowledge Professor Andreas Noack from the University of Applied Sciences Stralsund in Germany. Not only did he co-create the excellent tool \textit{Universal Radio Hacker} which was used extensively in this thesis. He also offered up a lot of his time in personally helping me when I initially felt way out of my depth with RF hacking by answering questions about the URH tool, RF communication in general, and figuring out how to capture good signals for this system.

Next, I would like to thank my girlfriend, Caroline, who had to hear me go on and on about radio waves and RF hacking for months, handle the stressful periods, for continuously proofreading the thesis, and for putting up with this whole situation during a pandemic. The same goes for my family.

I would, of course, like to thank my supervisor at Försvarsmakten. They gave me invaluable insights and expertise during this entire project. All the way from selecting what type of system to explore, to sharing their knowledge during the pentesting phase, to proofreading the final version. They also lent me more than enough of their time, meeting with me every week to discuss the thesis which I really appreciated.

Above all, I would like to thank my KTH supervisor Pontus Johnson. Before even starting this thesis, his excellent course Ethical Hacking (EN2720) opened my eyes to this entire field and was easily my favorite course at KTH. He also personally helped me get in contact with and recommended me to several organizations within the security industry during the search for a place to write my thesis as well as during my job hunt after graduation. During the thesis, Pontus also gave a lot of his time, answered questions, and gave excellent feedback and encouragement.

Lastly, a special thanks to KTH for these last five years!

\acknowlegmentssignature

% ---- Table of contents, etc ----
\fancypagestyle{plain}{}
\tableofcontents
\markboth{\contentsname}{}

\clearpage\listoffigures
\clearpage\listoftables
\clearpage\printglossary[type=\acronymtype, title={List of acronyms and abbreviations}]
