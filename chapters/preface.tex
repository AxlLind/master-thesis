\titlepage
\bookinfopage

\frontmatter
\setcounter{page}{1}

% ---- English Abstract ----
\begin{abstract}
\markboth{\abstractname}{}
% Keep in mind that most of your potential readers are only going to read your title and abstract. This is why it is important that the abstract give them enough information that they can decide is this document relevant to them or not. Otherwise the likely default choice is to ignore the rest of your document.
% A abstract should stand on its own, i.e., no citations, cross references to the body of the document, acronyms must be spelled out, …
% Write this early and revise as necessary. This will help keep you focused on what you are trying to do.

% Write an abstract (250 and 350 words) with the following components:
%  - What is the topic area? (optional) Introduces the subject area for the project.
%  - Short problem statement
%  - Why was this problem worth a Master’s thesis project? (i.e., why is the problem both significant and of a suitable degree of difficulty for a Master’s thesis project? Why has no one else solved it yet?)
%  - How did you solve the problem? What was your method/insight?
%  - Results/Conclusions/Consequences/Impact: What are your key results/conclusions? What will others do based upon your results? What can be done now that you have finished - that could not be done before your thesis project was completed? The presentation of the results should be the main part of the abstract.
\todo

\subsection*{Keywords}
% Choosing good keywords can help others to locate your paper, thesis, dissertation, … and related work.}
% Choose the most specific keyword from those used in your domain, see for example:
% ACM's Computing Classification System (2012) or
% (2014) IEEE Taxonomy. 
% Mechanics:
% - The first letter of a keyword should be set with a capital letter and proper names should be capitalized as usual.
% - Spell out acronyms and abbreviations.
% - Avoid "stop words" - as they generally carry little or no information.
% - List your keywords separated by commas (",").
% Since you should have both English and Swedish keywords - you might think of ordering them in corresponding order (i.e., so that the nth word in each list correspond) - thus it would be easier to mechanically find matching keywords.
\todo

\end{abstract}

% ---- Swedish Abstract ----
\selectlanguage{swedish}
\begin{abstract}
% All theses at KTH are required to have an abstract in both English and Swedish.
% If you are writing your thesis in English, you can leave this until the final version. If you are writing your thesis in Swedish then this should be done first – and you should revise as necessary along the way.
% If you are writing your thesis in English, then this section can be a summary targeted at a more general reader. However, if you are writing your thesis in Swedish, then the reverse is true – your abstract should be for your target audience, while an English summary can be written targeted at a more general audience.
% The Swedish sammanfattning need not be a literal translation of the english abstract.
\todo

\subsection*{Nyckelord}
\todo

\end{abstract}

\clearpage

\selectlanguage{english}
\section*{Acknowledgments}
\markboth{Acknowledgments}{}
I would like to thank Fredrik Heiding, PhD student within cybersecurity at KTH, for helping me procure the alarm system investigated in this thesis and navigating the KTH bureaucracy surrounding that process.

I would also like to acknowledge Professor Andreas Noack from the University of Applied Sciences Stralsund in Germany. Not only did he co-create the excellent tool \textit{Universal Radio Hacker} which was used extensively during this thesis. He also offered up a lot of his time in personally helping me when I was initially way out of my depth with RF hacking by answering questions about the URH tool, RF communication in general, and helping me figure out how to capture good signals.

Next, I would like to thank my girlfriend, Caroline, who had to hear me go on and on about radio waves and RF hacking for months, handle the stressful periods, for proofreading the thesis several times, and for putting up with this whole situation during a pandemic. The same goes for the rest of my family.

I would, of course, like to thank my supervisor at Försvarsmakten. They gave me invaluable insights and expertise during this entire project. All the way from selecting what type of system to explore, to sharing their knowledge during the pentesting phase, to proofreading the final version. They also lent me more than enough of their time, meeting with me every week to discuss the thesis which I really appreciated.

Above all, I would like to thank my KTH supervisor Pontus Johnson. Before even starting this thesis, his excellent course Ethical Hacking (EN2720) opened my eyes to this entire field and was easily my favorite course at KTH. He also personally helped me get in contact with and recommended me to several organizations within the security industry during the search for a place to write my thesis as well as during my job hunt after graduation. During the thesis, Pontus also gave a lot of his time, answered questions, and gave excellent feedback and encouragement.

Lastly, a special thanks to KTH for these last five years!

\acknowlegmentssignature

% ---- Table of contents, etc ----
\fancypagestyle{plain}{}
\tableofcontents
\markboth{\contentsname}{}

\clearpage\listoffigures
\clearpage\listoftables
\clearpage\printglossary[type=\acronymtype, title={List of acronyms and abbreviations}]
