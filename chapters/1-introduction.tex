\chapter{Introduction} \label{ch:intro}
Home automation and the number of connected devices in our homes have exploded in recent years. The number of \gls{IOT} devices especially has increased dramatically. It is predicted there will be about 38 billion of them by 2025 \cite{ieee-iot}. Many of \gls{IOT} devices are connected to the internet and that fraction is bound to increase given the rise of 5G technology. While these devices can do amazing things, everything from smart speakers to connected refrigerators, they are unfortunately hardly famous for their security. While this is well known in the IT-security community, the general non-tech-savvy consumer is perhaps not as aware of the security considerations when bringing an \gls{IOT} device into their home.

A type of connected device that has become increasingly common in people's homes are smart Home Alarm Systems. In fact, the global market for smart home alarm systems is expected to grow by 20\% in 2021 alone \cite{alarmsystem-market-analysis}. They protect your house from intruders by sounding an alarm when a suspected intrusion has occurred. Often a security central is immediately notified and security personnel sent to the site to investigate. These systems can be incredibly complex and can include multiple external peripherals like motion detectors, surveillance cameras, smoke detectors, etc. In recent years their scope and complexity has rapidly expanded even further. A modern smart home alarm system can now often control home automation systems like smart light bulbs and connected coffee machines, can interact with your smart speaker, and can even lock and unlock your door via smart locks. Additionally, they can be controlled remotely via mobile apps and web portals. These are undoubtedly useful features and undeniably these systems do provide protection against physical intrusion. However, one might wonder, given their recent rise in complexity, how secure these systems are against cyberattacks. How much of a focus is the cybersecurity of these systems is to the companies behind them? Given the large increase of features, have the vendors thought of all new emergent cybersecurity threats?

This thesis will examine the cybersecurity of a smart home alarm system from Securitas, called SecuritasHome. There were many aspects to consider when deciding what system to investigate, which is detailed in section \ref{ch:system:selection}. The SecuritasHome system comes with features such as alarming the system using a remote keypad and a four-digit pin, smoke detection, motion detection with a corresponding camera, and control of home automation devices (see section \ref{ch:system}). However, the main panel of the system, the \textit{brain of the system} so to speak, is connected to the internet via a local Ethernet cable as well as 3G telecommunication. If one were to comprise the security of this system there could be devastating consequences such as disarming the alarm, allowing an intruder to enter the house without setting triggering the alarm.

\section{Research question} \label{ch:intro:research-question}
This report will try and answer the following research question:

\begin{quote}
    \textit{Is the SecuritasHome Home Alarm System secure against cyberattacks?}
\end{quote}

In particular, this question can be broken down into two parts:

\begin{itemize}
    \item What vulnerabilities are present in the system?
    \item How can the vulnerabilities be exploited?
\end{itemize}

The security analysis presented in this thesis was performed on the following firmware versions. These were the latest versions at the time of writing (spring of 2021):
\begin{itemize}
    \item Alarm.com version: \texttt{193d}
    \item Climax Technology version: \texttt{HPGW-G 0.0.2.23F\\ BG\_U-ITR-F1-BD\_BL.A30.20181117}
\end{itemize}

\section{Objectives} \label{ch:intro:objectives}
The objective of this thesis is to assess the security of the SecuritasHome home alarm system. In essence, the objective in terms of the degree project is to assess whether or not the system can be considered secure from a computer security perspective. To achieve this a comprehensive security analysis was made of the system, to investigate which attack vectors the system is vulnerable to. Considering the large attack surface of the system in question, given its complexity and variety of features, some areas had to be delimited. More on this in section \ref{ch:intro:delimitations}.

From the perspective of the host organization, \textit{Försvarsmakten}, the objectives were to assess the security of home alarm systems in general, which have become increasingly common in Swedish homes. While these systems are generally considered effective against physical intrusion, less is sure about their security when it comes to cyberattacks. The host organization wanted a thorough investigation into the IT security of such a system.

\section{Methodology} \label{ch:intro:methodology}
During this thesis a seven-step penetration test methodology, as presented by \citeauthor{weidman2014} \cite{weidman2014}, was followed. This is detailed in section \ref{ch:method:pentest}. For the threat modeling phase of the project a threat modeling technique specialized for IoT systems presented by \citeauthor{guzman2017iot} \cite{guzman2017iot}, was used. Additionally, the OWASP top ten list of vulnerabilities for IoT systems \cite{owasp-iot-top10} was used to identify threats and used as an attack library, as well as the ETSI EN 303 645 standard \cite{etsi-iot-standard}. The methodology is explained in more detail in chapter \ref{ch:method}.

\section{Delimitations} \label{ch:intro:delimitations}
The system under consideration is very complex. It consists of many features, applications, and physical peripherals. As such, there is unfortunately not enough time within the scope of a degree project to exhaustively consider the full attack surface. Some things were also delimited due to legal reasons. As such the following major delimitations were done early in the project:
\begin{itemize}
    \item The external cloud servers, hosted by \textit{Alarm.com}. Legally, the security of these cannot be assessed without their permission.
    \item The mobile application, both the iOS and Android versions. This was delimited for two primary reasons, the major one being time and the other being the author not having easy access to an iOS device.
    \item The 3G wireless telecommunication. This was primarily due to the custom hardware required and the general security of this encrypted protocol \citeauthor{koien2003security}.
    \item The security of additional peripherals not included in the starter kit, see \ref{ch:system:hardware}, as well as Z-wave peripherals.
    \item Any attack requiring \textit{physical access} to the system. The goal of the system is to prevent physical access to the home. All components of the system are located inside so if an attacker has physical access to the system then the goal has already failed. Therefore these types of attacks were not deemed interesting to explore. Note that this does not exclude attacks that only require physical proximity, like being outside the door for example.
\end{itemize}

\section{Structure of the thesis} \label{ch:intro:structure}
This report is structured into the following chapters:
\begin{itemize}
    \item Chapter \ref{ch:intro} (\nameref{ch:intro}) gives an introduction to the thesis area and research question. Additionally, delimitations of the project are listed.
    \item Chapter \ref{ch:method} (\nameref{ch:method}) gives a thorough explanation of the methodology of this thesis. First, the general penetration testing methodology applied in the project is explained and lastly, the threat modeling technique is explained.
    \item Chapter \ref{ch:system} (\nameref{ch:system}) explains the system under consideration in detail, e.g all hardware and software components of the system, as well how the system was selected and an overview of the companies behind the system.
    \item Chapter \ref{ch:related-work} (\nameref{ch:related-work}) lists identified related work to this thesis. This, among others, includes a security analysis of a system based on similar hardware and partially the same firmware, as well as talks on \gls{RF} hacking.
    \item Chapter \ref{ch:threat-model} (\nameref{ch:threat-model}) presents a threat model of the system, more on the threat modeling technique used in section \ref{ch:method:threat-modeling}.
    \item Chapter \ref{ch:pentesting} (\nameref{ch:pentesting}) explains all penetration tests performed against the system, potential background information about the attack, the result of the test, as well as a discussion of the consequences and mitigations of them.
    \item Chapter \ref{ch:discussion} (\nameref{ch:discussion}) contains a discussion about the validity and efficiency of the methodology used, a discussion about the results from chapter \ref{ch:pentesting}, as well as a mandated section on the sustainability and ethics of the thesis.
    \item Chapter \ref{ch:conclusion} (\nameref{ch:conclusion}) concludes the thesis by presenting the conclusions of the report, to what extent the system can be considered \textit{secure}, as well as a discussion about future work that could be done on examining the system's security.
\end{itemize}
