\chapter{Introduction} \label{ch:intro}
Home automation and the number of connected devices in our home has exploded in recent years. The number of \gls{IOT} devices especially have increased dramatically. It is predicted there will be about 38 billion of them by 2025 \cite{ieee-iot}. Many of \gls{IOT} devices are connected to the internet and that fraction is bound to increase given the rise of 5G technology. While these devices can do amazing things, everything from smart speakers to connected refrigerators, they are hardly known for their security. While this is well known in the IT-security community, the general non-tech-savvy consumer are perhaps not as aware of the security considerations when bringing an \gls{IOT} device into their home.

A type of connected device that has become increasingly common in people's homes and which are becoming increasingly complex are smart Home Alarm Systems. They protect your house from intruders by sounding an alarm when an suspected intrusion has occurred. Often a security central is immediately notified and security personnel sent to the site to investigate. These systems can be incredibly complex and can include multiple external peripherals like motion detectors, surveillance cameras, smoke detectors, etc. In recent years their scope have expanded further and can now often control home automation systems like smart light bulbs, connected coffee machines, and even the lock to your door. Additionally, they can often be controlled remotely via mobile apps and web-portals. While these are undoubtedly useful features and undeniably provide some protection against physical intrusion one might wonder how secure these systems are against cyberattacks and much of a focus the cybersecurity of these systems is to the companies behind them. Given the large increase of features, have the vendors thought of all cybersecurity threats?

This thesis will examine the cybersecurity of a smart Home Alarm System from SecuritasHome. The SecuritasHome system is a Home Alarm System with features such as alarming the system using a remote keypad and a four-digit pin, smoke detection, motion-detection with a corresponding camera, and control of home automation devices (see section \ref{ch:system}). However, the main panel of the system (the \textit{brain of the system}) is connected to the internet, both the local network and the mobile 3G network. If one were to comprise the security of this system there could be devastating consequences, such as disarming the alarm and entering the house without setting it off.

\section{Research question} \label{ch:intro:research-question}
This report will try and answer the following research question:

\begin{quote}
    \textit{Is the SecuritasHome Home Alarm System secure against cyberattacks?}
\end{quote}

\noindent In particular, this question can be broken down into two parts:

\begin{itemize}
    \item What vulnerabilities are present in the system?
    \item How can the vulnerabilities be exploited?
\end{itemize}

\noindent The security analysis presented in this thesis was performed on the following firmware versions. These were the latest versions at the time, the spring of 2021:
\begin{itemize}
    \item Alarm.com firmware version: \texttt{193d}
    \item Climax Technology firmware version: \texttt{HPGW-G 0.0.2.23F\\ BG\_U-ITR-F1-BD\_BL.A30.20181117}
\end{itemize}

\section{Background} \label{ch:intro:background}
[TODO]

\section{Objectives} \label{ch:intro:objectives}
The objective of this thesis is to asses the security of the SecuritasHome Home Alarm System. In essence, the objective in terms of the degree project is to asses whether or not the system can be considered secure from an computer-security perspective. To achieve this a comprehensive security audit was made to the system, to investigate which attack vectors the system is vulnerable to. Considering the large attack surface of the system in question, given it's complexity and variety of features, some areas where delimited. More on this in \ref{ch:intro:delimitations}.

From the perspective of the host organization, \textit{Försvarsmakten}, the objectives were to asses the security of home alarm systems in general, which have become increasingly common. While these systems are generally considered effective against physical intrusion, less is sure about their security when it comes to cyberattacks. The host organization wanted a thorough investigation into the IT-security of such a system.

\section{Methodology} \label{ch:intro:methodology}
[TODO]

\section{Delimitations} \label{ch:intro:delimitations}
The system under consideration of this thesis is very complex. It consists many features, applications, and physical peripherals. As such, there is unfortunately not enough time within the scope of a degree project to exhaustively consider the full attack surface. Some things were also delimited due to legal reasons. As such the following major delimitations were done early in the project:

\begin{itemize}
    \item The external cloud servers, hosted by \textit{Alarm.com}. Legally, the security of these could not be assesed.
    \item The 3G wireless telecommunication. This was primarily due to the custom hardware required and the well-known security of this encrypted protocol.
    \item The security of additional peripherals not included in the starter-kit, see \ref{ch:system:hardware}.
    \item The iOS mobile application. This was delimited for two primary reasons, the major one being time and the other being the author not having easy access to an iOS device.
    \item The Android mobile application. [Note: Maybe?]
\end{itemize}

\noindent Additionally, more fine-grained delimitations were done during the exploratory phase of the project. More on this in [REF].

\section{Structure of the thesis} \label{ch:intro:structure}
This report is structured in to the following chapters:
\begin{itemize}
    \item Chapter \ref{ch:intro} gives an introduction into the thesis and research question, as well as a brief introduction into the background of the subject area.
    \item TODO...
\end{itemize}
