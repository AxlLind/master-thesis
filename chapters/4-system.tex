\chapter{System Under Consideration} \label{ch:system}
The following chapter gives a thorough explanation of the system under consideration of this thesis. The system in question is the \textit{SecuritasHome startpaket}, which is a home alarm system from Securitas. This starter-kit includes multiple hardware components, as well as access to software portals and mobile applications to control the system.

\section{Components and Software} \label{ch:system:components}
This section describes all the components and software of the system. Initially, all hardware components are described and their functionality. Lastly, all software components of the system are described. Note that the system supports many additional hardware components. The ones outlined below are only the ones part of the starter-kit.

\subsection{Hardware components} \label{ch:system:hardware}
The SecuritasHome starter-kit contains five hardware components, see figure \ref{fig:hardware-components}. These are described below. Note that the system supports many additional components, like smart locks for example. However, only the components included in the starter-kit are covered in this thesis.
\begin{figure}[!ht]
    \centering
    \begin{subfigure}[t]{0.33\textwidth}
        \includegraphics[height=2.15in]{images/main-panel.png}
        \caption{Main Panel}
        \label{fig:main-panel}
    \end{subfigure}%
    ~
    \begin{subfigure}[t]{0.33\textwidth}
        \includegraphics[height=2.15in]{images/keypad.png}
        \caption{Remote Keypad}
        \label{fig:remote-keypad}
    \end{subfigure}%
    ~
    \begin{subfigure}[t]{0.33\textwidth}
        \includegraphics[height=2.15in]{images/camera.png}
        \caption{Motion Detection Camera}
        \label{fig:motion-camera}
    \end{subfigure}
    
    \begin{subfigure}[t]{0.33\textwidth}
        \includegraphics[height=2.15in]{images/door-contact.png}
        \caption{Door Contact Sensor}
        \label{fig:door-contact}
    \end{subfigure}%
    ~
    \begin{subfigure}[t]{0.33\textwidth}
        \includegraphics[height=2.15in]{images/smoke-detector.png}
        \caption{Smoke Detector}
        \label{fig:smoke-detector}
    \end{subfigure}
    \caption{The hardware components of the system}
    \label{fig:hardware-components}
\end{figure}
\subsubsection{Main Panel}
\textbf{Model number:} HSGW-G8-3G/LTE-ZW-F1 433/868 \\
\textbf{FCCID:} GX9HSGWF1919 \\
The main panel, see figure \ref{fig:main-panel}, is the \textit{"brains"} of the system so to speak. It handles communication with all other hardware devices as well as external servers. Through radio wave communication it talks to the other hardware peripherals of the system. It uses 3G telecommunication to talk to the external servers.

\subsubsection{Remote Keypad}
\textbf{Model number:} KPT-23-EL-F1 \\ % could also be KPT-23N-EL-F1, not sure
\textbf{FCCID:} GX9KPF1 \\ % This says KPF but it looks the same..
The remote keypad is a 16 button keypad used to arm and disarm the system using a personal 4 digit pin. See figure \ref{fig:remote-keypad}. This device talks to the main panel over radiowave communication.

\subsubsection{Motion Detection Camera}
\textbf{Model number:} VST-862-F1 \\
\textbf{FCCID:} GX9862 \\
This device, see figure \ref{fig:motion-camera}, features an infra-red sensor to detect motion, and a camera to survey the location. When triggered the device takes two pictures which are sent to the main panel. It is not a surveillance camera, meaning it does not continuously take pictures. The camera is only active when motion is detected and the alarm is triggered, presumably to save power.

\subsubsection{Door Contact Sensor}
\textbf{Model number:} DC-23-F1 \\
\textbf{FCCID:} GX9DC23 \\
This device, see figure \ref{fig:door-contact}, senses when a door or window is opened. A small external magnet is placed on the door/window close to the device. When these are separated the device is triggered and communicates with the main panel over radio wave communication.

\subsubsection{Smoke Detector}
\textbf{Model number:} SD-8EL \\
\textbf{FCCID:} GX9SD8ELF1919 \\
This device is a smoke detector, see figure \ref{fig:smoke-detector}. It communicated with the main panel over radio-waves and also includes a siren which triggers when it detects smoke.

\subsection{Software} \label{ch:system:software}
\subsubsection{Web portal}

\subsubsection{Mobile application}

\subsubsection{Local web admin page}

\section{The companies behind the system}
This section covers the structure of the three major companies behind the SecuritasHome Home Alarm System.

While the system is sold by SecuritasHome, they actually have little to do with the actual hardware and software components of the platform, see figure \ref{fig:company-structure}. The hardware, related firmware, and radio wave protocol are manufactured and produced by a Taiwanese company called \textit{Climax Technology}. They are a major manufacturer of hardware in the security industry, producing everything from smart home alarm systems, and garage door openers, to smart medical accessories for seniors. The software, like the web portal and mobile applications as well as some additional firmware, are developed by an American company called \textit{Alarm.com}. They are strictly a B2B (business to business) company, meaning they do not sell or market their product directly to consumers. Instead they outsource the sale and advertisement of the system to other companies, one of them being Securitas. SecuritasHome merely sell, advertise, and put their brand on the product. Their main contribution to the system is in terms of real-time response to an alarm, connection to an alarm-central, and sending security personnel to respond to an active alarm breach. Consequently, when considering the cybersecurity of the system, Securitas is not strictly relevant.
\begin{figure}[!ht]
    \centering
    \includegraphics[width=0.5\textwidth]{images/company-structure.png}
    \caption{The companies behind the system.}
    \label{fig:company-structure}
\end{figure}
