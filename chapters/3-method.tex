\chapter{Method}\label{ch:methods}
% This chapter is about Engineering-related content, Methodologies and Methods. Use a self-explaining title. The contents and structure of this chapter will change with your choice of methodology and methods.

% Describe the engineering-related contents (preferably with models) and the research methodology and methods that are used in the degree project. 
% Give a theoretical description of the scientific or engineering methodology are you going to use and why have you chosen this method. What other methods did you consider and why did you reject them.

% In this chapter, you describe what engineering-related and scientific skills you are going to apply, such as modeling, analyzing, developing, and evaluating engineering-related and scientific content. The choice of these methods should be appropriate for the problem . Additionally, you should be consciousness of aspects relating to society and ethics (if applicable). The choices should also reflect your goals and what you (or someone else) should be able to do as a result of your solution - which could not be done well before you started.

% The purpose of this chapter is to provide an overview of the research method used in this thesis. Section~\ref{sec:researchProcess} describes the research process. Section~\ref{sec:researchParadigm} details the research paradigm. Section~\ref{sec:dataCollection} focuses on the data collection techniques used for this research. Section~\ref{sec:experimentalDesign} describes the experimental design. Section~\ref{sec:assessingReliability} explains the techniques used to evaluate the reliability and validity of the data collected. Section~\ref{sec:plannedDataAnalysis} describes the method used for the data analysis. Finally, Section~\ref{sec:evaluationFramework} describes the framework selected to evaluate xxx.

% Vilka vetenskapliga eller ingenjörsmetodik ska du använda och varför har du valt den här metoden. Vilka andra metoder gjorde du överväga och varför du avvisar dem.
% Vad är dina mål? (Vad ska du kunna göra som ett resultat av din lösning - vilken inte kan göras i god tid innan du började)
% Vad du ska göra? Hur? Varför? Till exempel, om du har implementerat en artefakt vad gjorde du och varför? Hur kommer ditt utvärdera den.
% Syftet med detta kapitel är att ge en översikt över forsknings metod som används i denna avhandling. Avsnitt~\ref{sec:researchProcess} beskriver forskningsprocessen. Avsnitt~\ref{sec:researchParadigm} detaljer forskningen paradigm. Avsnitt~\ref{sec:dataCollection} fokuserar på datainsamling tekniker som används för denna forskning. Avsnitt~\ref{sec:experimentalDesign} beskriver experimentell design. Avsnitt~\ref{sec:assessingReliability} förklarar de tekniker som används för att utvärdera tillförlitligheten och giltigheten av de insamlade uppgifterna. Avsnitt~\ref{sec:plannedDataAnalysis} beskriver den metod som används för dataanalysen. Slutligen, Avsnitt~\ref{sec:evaluationFramework} beskriver ramverket valts för att utvärdera xxx.
[TODO]

\section{Research Process} \label{sec:researchProcess}
[TODO]

\section{Research Paradigm} \label{sec:researchParadigm}
% Exempelvis Positivistisk (vad/hur fungerar det?) kvalitativ fallstudie med en deduktivt (förbestämd) vald ansats och ett induktivt(efterhand uppstår dataområden och data) insamlade av data och erfarenheter.
[TODO]

\section{Data Collection} \label{sec:dataCollection}
% This should also show that you are aware of the social and ethical concerns that might be relevant to your data collection method.)
[TODO]

\section{Experimental design/Planned Measurements} \label{sec:experimentalDesign}
[TODO]

\subsection{Test environment/test bed/model}
% Describe everything that someone else would need to reproduce your test environment/test bed/model/…
[TODO]

\subsection{Hardware/Software to be used}
[TODO]

\section{Assessing reliability and validity of the data collected} \label{sec:assessingReliability}
[TODO]

\subsection{Validity of method}
% How will you know if your results are valid?
[TODO]

\subsection{Reliability of method}
% How will you know if your results are reliable?
[TODO]

\subsection{Data validity}
% Hur vet du om dina resultat är giltiga? Har ditt resultat mäta rätta?
[TODO]

\subsection{Reliability of data}
% Hur vet du om dina resultat är tillförlitliga? Hur bra är dina resultat?
[TODO]

\section{Planned Data Analysis} \label{sec:plannedDataAnalysis}
[TODO]


\subsection{Data Analysis Technique}
[TODO]

\subsection{Software Tools}
[TODO]

\section{Evaluation framework} \label{sec:evaluationFramework}
[TODO]

\section{System documentation}
% If this is going to be a complete document consider putting it in as an appendix, then just put the highlights here.

% Med vilka dokument och hur skall en konstruerad prototyp dokumenteras? Detta blir ofta bilagor till rapporten och det som problemägaren till det ursprungliga problemet (industrin) ofta vill ha.Bland dessa bilagor återfinns ofta, och enligt någon angiven standard, kravdokument, arkitekturdokument, designdokumnet, implementationsdokument, driftsdokument, testprotokoll mm.
[TODO]