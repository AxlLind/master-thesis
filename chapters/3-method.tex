\chapter{Method} \label{ch:method}
The following chapter described the method used in this report. It is based on a seven step process to penetration testing by \textcite{weidman2014}. In the first part, this method is described. That is followed by how each of these seven steps were carried out on this thesis. Additionally, for the threat modeling phase a method outlined by \textcite{guzman2017iot} was used. This threat modeling process is described below.

\section{Penetration Testing methodology} 
In their book \citetitle{weidman2014}, \citeauthor{weidman2014} details a seven step process for penetration testing \cite{weidman2014}. This section firstly gives a brief description of all seven steps, and lastly outlines how each step was performed in this thesis.

Included in \citeauthor{weidman2014}'s method for penetration testing are the following seven steps:
\begin{enumerate}
    \item \textit{Pre-engagement}. This step involves communicating with the party that ordered the penetration test to be done. The goal of this step is to make sure both parties are on the same page and understanding of how the tests will be done. Things to agree upon, according to \citeauthor{weidman2014}, are scope, testing window, and clear authorization from the other party that you are legally allowed to assess the security of their system.
    \item \textit{Information Gathering}. This step includes what is known as \gls{OSINT}. \gls{OSINT} is the process of using publicly available sources of information to gather information about the system in question. These include search engines like Google, news articles, public government data, academic papers, etc \cite{steele2007open}. One might also use port scanners like Nmap\footnotelink{https://nmap.org/}{2021-03-29} and other application scanners to gather information about the system. Additionally, one might sniff the network traffic of the application to gain an understanding of it's behavior.
    \item \textit{Threat Modeling}. This step involves mapping out the components of the system, based on the information from the previous step. From that, you think of potential attacks and vulnerabilities of the system, their potential impact, and likelihood of success. More on this in \ref{ch:method:threat-modeling}.
    \item \textit{Vulnerability Analysis}. This step involves actively pentesting the system to discover vulnerabilities. This can be done for example through manually probing the system or by using vulnerability scanners like \textit{Metasploit}\footnotelink{https://www.metasploit.com/}{2021-03-29}, \textit{Burp Suite}\footnotelink{https://portswigger.net/burp}{2021-03-29}, or \textit{Nessus}\footnotelink{https://www.tenable.com/products/nessus}{2021-03-29}.
    \item \textit{Exploitation}. This step involves exploiting the vulnerabilities discovered in the previous step. By exploiting these, the goal is to perform some malicious act on the system, see \ref{ch:method:stride}.
    \item \textit{Post Exploitation}. After a successful exploit, this step involves analyzing the consequences of it. If the exploit involves access to a machine one might investigate the file system, look for possibilities of privilege escalation, etc. One asks how sever this successful exploit is to the overall security of the system.
    \item \textit{Reporting}. This final step involves summarizing the findings to the interested party. Crucially, if the findings are to be publicized one should adhere to the principle of responsible disclosure.
\end{enumerate}
What follows is a description of how each step above was applied in this thesis.

\subsection{Pre-engagement}
According to \citeauthor{weidman2014}'s method, the pre-engagement step is done in collaboration with the client. In this project, however, there is no clear client except the author and perhaps KTH and Försvarsmakten. The scope and expectations were continuously discussed during the course of the project. The companies behind the system (see \ref{fig:company-structure}) were not informed of the security analysis until after the project was finalized. Securitas, the seller of the system was contacted multiple times via their support to verify the legality of security testing the system.

\subsection{Information Gathering}
The information gathering phase was done in several steps, the first one being \gls{OSINT}. Initially, the model number of all devices were gathered from either physical labels on the peripherals or from Securitas' website\footnotelink{https://www.securitashome.se/}{2021-03-29}. Using the search engine Google, the devices and their manufacturer \textit{Climax Technology} were quickly identified. From their website much more information about the system could be found, such as how the peripherals communicate and their \gls{RF} protocol\footnotelink{https://www.climax.com.tw/new/f1-features-new.php}{2021-03-29}. An additional resource was finding each component's FCC ID\footnotelink{https://www.fcc.gov/oet/ea/fccid}{2021-03-30}, from which one can find user manuals submitted to the FCC agency, official testing documentation, and more via \textit{fccid.io}.

\subsection{Threat Modeling} \label{ch:method:threat-modeling}
[TODO]

\subsection{Vulnerability Analysis}
[TODO]

\subsection{Exploitation}
[TODO]

\subsection{Post Exploitation}
[TODO]

\subsubsection{Reporting}
[TODO]

\section{The STRIDE model} \label{ch:method:stride}
STRIDE is a model of threats used to identify threats to the cybersecurity of an IT system \cite{stride}. It was initially developed by Praerit Garg and Loren Kohnfelder at Microsoft as part of their threat modeling technique. It is a widely used mnemonic in the security industry to aid in recognizing threats\footnotelink{https://docs.microsoft.com/en-us/azure/security/develop/threat-modeling-tool-threats}{2021-03-29}. The following are the six properties that make up the acronym and a description of each:
\begin{itemize}
    \item \textbf{S}poofing identity. This means impersonating the identity of another user or component of the system. One example is obtaining a users login credentials and posing as that user.
    \item \textbf{T}ampering with data. This means modifying data in the system in a malicious manner that you were perhaps not meant to modify. An example is unauthorized modifications to data stored in a database.
    \item \textbf{R}epudiation. This means being able to claim you did not perform a certain action. An example would be to somehow be able to claim a transaction did not go through, thereby illegally receiving additional payment.
    \item \textbf{I}nformation disclosure. This means exposing information or data to users who are not meant to have access to it. An example is a database leak.
    \item \textbf{D}enial of service. This means denying access to a service. An example is making a web server unavailable to users by hitting it with heavy traffic, perhaps via a distributed \gls{DOS} attack.
    \item \textbf{E}levation of privilege. This means an unprivileged user gaining privileged access to the system, allowing them access to features of the system they were not meant to. An example is exploiting a vulnerability in a program or kernel to get access as a more privileged user on the machine (like root).
\end{itemize}
