\usepackage[perpage,para,symbol]{footmisc}
\usepackage[acronym, section=section, nonumberlist, nomain, nopostdot]{glossaries}
% \usepackage[style=numeric,sorting=none,backend=biber]{biblatex}

\usepackage{tikz}
\usetikzlibrary{arrows,decorations.pathmorphing,backgrounds,fit,positioning,calc,shapes}
\usepackage{pgfmath}      % --math engine
\usepackage{rotating}     % For text rotating
\usepackage{array}        % For table wrapping
\usepackage{graphicx}     % Support for images
\usepackage{float}        % Support for more flexible floating box positioning
\usepackage{mdwlist}      % various list-related commands
\usepackage{setspace}     % For fine-grained control over line spacing
\usepackage{listings}     % For source code listing
\usepackage{bytefield}    % For packet drawings
\usepackage{tabularx}     % For simple table stretching
\usepackage{multirow}     % Support for multirow colums in tables
\usepackage{url}          % Support for breaking URLs
\usepackage{hyperref}
\usepackage[all]{hypcap}  % prevents an issue related to hyperref and caption linking
%% setup hyperref to use the darkblue color on links
\hypersetup{colorlinks,breaklinks,
            linkcolor=darkblue,urlcolor=darkblue,
            anchorcolor=darkblue,citecolor=darkblue}

%% Some definitions of used colors
\setlength {\marginparwidth }{2cm} % leave some extra space for todo notes
\hyphenpenalty=15000\tolerance=1000 % to reduce hyphenation
\definecolor{darkblue}{rgb}{0.0,0.0,0.3} %% define a color called darkblue
\definecolor{darkred}{rgb}{0.4,0.0,0.0}
\definecolor{red}{rgb}{0.7,0.0,0.0}
\definecolor{lightgrey}{rgb}{0.8,0.8,0.8} 
\definecolor{grey}{rgb}{0.6,0.6,0.6}
\definecolor{darkgrey}{rgb}{0.4,0.4,0.4}
\definecolor{aqua}{rgb}{0.0, 1.0, 1.0}

\usepackage{listings}
\usepackage[cache=false]{minted} %% For source code highlighting
\usemintedstyle{borland}

\usepackage{csquotes} % Recommended by biblatex

% set the chapter header
\renewcommand{\chaptermark}[1]{\markboth{#1}{}}